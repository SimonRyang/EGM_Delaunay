%2multibyte Version: 5.50.0.2960 CodePage: 1251
%\renewcommand{\baselinestretch}{1.5}
% Helps LaTeX put figures where YOU want
%\renewcommand{\topfraction}{0.8}
% 90% of page top can be a float
%\renewcommand{\bottomfraction}{0.8}
% 90% of page bottom can be a float
%\renewcommand{\textfraction}{0.2}
% only 10% of page must to be text
%\input{tcilatex}
%\input{tcilatex}
%\input{tcilatex}
%\input{tcilatex}
%\input{tcilatex}
%\input{tcilatex}
%\renewcommand{\labelenumi}{\Alph{enumi})}
%\renewcommand{\labelenumii}{\arabic{enumii}.}
%\documentclass[a4paper,12pt,ngerman]{article}
%\usepackage[abbr,agsmcite,agsm,round]{harvard}
%\pagenumbering{arabic}
%\renewcommand\harvardand{and}
%\renewcommand{\cite}{\citeasnoun}
%\renewcommand{\harvardand}{and}
%\setlength{\oddsidemargin}{0cm}
%\setlength{\textwidth}{15cm}
%\setlength{\textheight}{23cm}
%\setlength{\topmargin}{-1cm}
%\renewcommand{\thesection}{\Roman{section}}
%\renewcommand{\thesubsection}{\Alph{subsection}}
%\newenvironment{proof}[1][Proof]{\noindent \textbf{#1.} }{\  \rule{0.5em}{0.5em}}
%\usepackage[left=1.5in,right=1.5in,top=1.5in,bottom=1.5in]{geometry}

\documentclass[a4paper,12pt]{article}%
\usepackage{amsfonts}
\usepackage{amssymb}
\usepackage{amsmath}
\usepackage{eurosym}
\usepackage{latexsym}
\usepackage{graphicx}
\usepackage{longtable}
\usepackage{portland}
\usepackage{lscape}
\usepackage[onehalfspacing]{setspace}
\usepackage{footmisc}
\usepackage{fancyhdr}
\usepackage{hyphenat}
\usepackage{rotating}
\usepackage[USenglish]{babel}
\usepackage{array}
\usepackage{tabularx}
\usepackage{chicago}
\usepackage{theorem}
\usepackage{epstopdf}
\usepackage{makeidx}

\usepackage[left=1in,right=1in,top=1in,bottom=1in]{geometry}%
\setcounter{MaxMatrixCols}{30}
%TCIDATA{OutputFilter=latex2.dll}
%TCIDATA{Version=5.00.0.2606}
%TCIDATA{Codepage=1251}
%TCIDATA{CSTFile=Deutscher LaTeX Article.cst}
%TCIDATA{LastRevised=Thursday, August 29, 2013 16:07:24}
%TCIDATA{<META NAME="GraphicsSave" CONTENT="32">}
%TCIDATA{<META NAME="PrintViewPercent" CONTENT="100">}
%TCIDATA{<META NAME="SaveForMode" CONTENT="1">}
%TCIDATA{BibliographyScheme=BibTeX}
%TCIDATA{Language=American English}
%TCIDATA{PageSetup=43,43,71,71,0}
%TCIDATA{Counters=arabic,1}
%TCIDATA{AllPages=
%H=36
%F=36
%}
\newtheorem{theorem}{Theorem}
\newtheorem{acknowledgement}[theorem]{Acknowledgement}
\newtheorem{algorithm}[theorem]{Algorithm}
\newtheorem{assum}{Assumption}
\newtheorem{axiom}[theorem]{Axiom}
\newtheorem{case}[theorem]{Case}
\newtheorem{claim}[theorem]{Claim}
\newtheorem{conclusion}[theorem]{Conclusion}
\newtheorem{condition}[theorem]{Condition}
\newtheorem{conjecture}[theorem]{Conjecture}
\newtheorem{corollary}[theorem]{Corollary}
\newtheorem{criterion}[theorem]{Criterion}
\newtheorem{definition}{Definition}
\newtheorem{example}[theorem]{Example}
\newtheorem{exercise}[theorem]{Exercise}
\newtheorem{lemma}[theorem]{Lemma}
\newtheorem{notation}[theorem]{Notation}
\newtheorem{problem}[theorem]{Problem}
\newtheorem{proposition}{Proposition}
\newtheorem{remark}[theorem]{Remark}
\newtheorem{solution}[theorem]{Solution}
\newtheorem{summary}[theorem]{Summary}
\newtheorem{observation}{Observation}
\newtheorem{result}{Result}
\begin{document}

\title{Endogenous Grids in Higher Dimensions: Delaunay Interpolation and Hybrid
Methods\thanks{We thank Johannes Brumm, Christopher Carroll, Michael Reiter
and seminar participants at University of Cologne, the 2012 CEF and the
Cologne Macroeconomic Workshop 2012 for helpful comments.}}
\author{Alexander Ludwig\thanks{CMR \& FiFo, University of Cologne; MEA; Netspar;
Albertus-Magnus-Platz; 50923 K\"oln; Germany; E-mail:
ludwig@wiso.uni-koeln.de}
\and Matthias Sch\"{o}n\thanks{CMR, University of Cologne; Albertus-Magnus-Platz;
50923 K\"{o}ln; Germany; E-mail: m.schoen@wiso.uni-koeln.de}}
\date{\today }
\maketitle

\begin{abstract}
This paper investigates extensions of the method of endogenous grid-points
(ENDGM) introduced by Carroll (2006) to higher dimensions with more than one
continuous endogenous state variable. We compare three different categories of
algorithms: (i) the conventional method with exogenous grids (EXGM), (ii) the
pure method of endogenous grid-points (ENDGM) and (iii) a hybrid method
(HEGM). ENDGM comes along with Delaunay interpolation on irregular grids.
Comparison of methods is done by evaluating speed and accuracy. We find that
HEGM and ENDGM both dominate EXGM. The choice between HEGM and ENDGM depends
on the number of dimensions and the number of grid-points in each dimension.
With less than~150 grid-points in each dimension ENDGM is faster than HEGM,
and vice versa. For a standard choice of~$20$ to~$40$ grid-points in each
dimension, ENDGM is~$1.6$ to~$1.8$ times faster than HEGM. \newline\textit{JEL
Classification: C63, E21.\newline Keywords: Dynamic Models, Numerical
Solution, Endogenous Gridpoints Method, Delaunay Interpolation}

\end{abstract}

\newpage\pagenumbering{arabic} \renewcommand{\thefootnote}{\arabic{footnote}} \setcounter{footnote}{0}

\section{Introduction}

Dynamic models of equilibrium in discrete time are workhorse models in
Economics. However, most of these models do not have an analytic closed form
solution and equilibria have to be approximated numerically. Numerous
procedures have been developed in the literature, cf.~%
%TCIMACRO{\TeXButton{citeN{judd98}}{\citeN{judd98}}}%
%BeginExpansion
\citeN{judd98}%
%EndExpansion
,~%
%TCIMACRO{\TeXButton{citeN{miranda04}}{\citeN{miranda04}}}%
%BeginExpansion
\citeN{miranda04}%
%EndExpansion
. If the problem is differentiable, a popular approach is to use first-order
methods, i.e., to iterate on first-order conditions. An important contribution
in this literature is~%
%TCIMACRO{\TeXButton{citeN{carroll06}}{\citeN{carroll06}} }%
%BeginExpansion
\citeN{carroll06}
%EndExpansion
who introduces the method of endogenous grid points (ENDGM). In comparison to
the method of exogenous grid-points (EXGM), ENDGM greatly enhances
computational speed because part of the problem can be computed in closed form.

This paper investigates extensions of Carroll's method to dynamic problems
with more than one continuous endogenous state variable. We highlight an
important drawback of ENDGM in higher dimensions which is due to the
endogenously computed states. The resulting state grid is not rectangular,
i.e., grid-points are irregularly distributed in the space. In consequence,
even linear interpolation is much more costly than for conventional
rectangular grids. Hence, there exists a fundamental trade-off between EXGM
and ENDGM in higher dimensions. On the one hand, EXGM requires the use of
numerical routines throughout whereas ENDGM computes solutions to first-order
conditions in closed form. On the other hand, interpolation in EXGM is on
regular grids and therefore simple. Interpolation in ENDGM on irregular grids
is much more complex. We solve this complex interpolation by Delaunay
triangulation. Delaunay interpolation originally coming from the field of
geometry is not commonly adopted in Economics. It was only recently introduced
to the field by~\citeN{brumm10}.

In addition to EXGM and ENDGM, we consider a third algorithm, a hybrid method
of exogenous and endogenous grid-points (HEGM). HEGM uses endogenous grids in
some but not all dimensions. The trade-off between HEGM and ENDGM is therefore
between costly routines in some dimensions vis-\`{a}-vis analytical solutions
in all dimensions but a more complex interpolation.

To analyze and to compare these methods we use a simple human capital model.
This model features two endogenous state variables, financial assets and human
capital. Evaluation of methods in this two dimensional setup is done by
comparing speed and accuracy of the different approaches.

Our main finding is that HEGM and ENDGM both dominate EXGM. They are both
substantially faster. The choice between HEGM and ENDGM depends on the number
of grid-points in each dimension. For a relatively low number of grid-points,
ENDGM is advantageous and vice versa for HEGM. We discuss limitations of ENDGM
and HEGM which are both only applicable for specific problems at hand.
Furthermore, we argue that the relative advantage of HEGM decreases in higher
dimensions (higher than two).

To the best of our knowledge ENDGM in higher dimensions is not yet fully
understood. Our paper is an important contribution to fill this gap. Related
work by~%
%TCIMACRO{\TeXButton{citeN{krueger07}}{\citeN{krueger07}} }%
%BeginExpansion
\citeN{krueger07}
%EndExpansion
and~%
%TCIMACRO{\TeXButton{citeN{Barillas07}}{\citeN{Barillas07}} }%
%BeginExpansion
\citeN{Barillas07}
%EndExpansion
extends ENDGM to problems with two control variables but just one endogenous
state variable.
%TCIMACRO{\TeXButton{citeN{Hintermaier10}}{\citeN{Hintermaier10}} }%
%BeginExpansion
\citeN{Hintermaier10}
%EndExpansion
incorporate two endogenous state variables in a durable goods model and apply
a version of HEGM. Their algorithm exploits the fact that durable goods
consumption is zero at the end of life. In a human capital model such as ours
this is not possible because the human capital stock is neither zero nor known
at the end of life. Hence our version of HEGM differs from the
%TCIMACRO{\TeXButton{citeN{Hintermaier10}}{\citeN{Hintermaier10}} }%
%BeginExpansion
\citeN{Hintermaier10}
%EndExpansion
approach.

Our analysis proceeds as follows. Section 2 presents the simple human capital
setting on which we base the evaluation of methods. Section 3 introduces the
main features of the methods under evaluation, the method of exogenous
grid-points, the pure method of endogenous grid-points and the hybrid method
of exogenous and endogenous grid-points. Section 4 presents results according
to speed and accuracy of all three methods. Section 5 concludes. Additional
material is contained in an appendix.

\section{General Framework}

We develop a human capital model which allows us to illustrate and compare
three approaches to solve dynamic models with two endogenous states using
first-order methods. In addition to financial assets there is another
endogenous state, a human or health capital stock (we will use both
interpretations interchangeably). Human capital can be accumulated over time
and is produced with a nonlinear production function. For expositional
purposes we keep the model simple. Of course, the underlying trade-off between
solution methods will also hold in more complex problems.

\subsection{A Simple Human Capital Model}

\label{ss:simplehkmodel}

We consider a setup in which the only risk is the risk to survive to the next
period. A risk averse agent with maximum time horizon $T$, $T=\infty$
possible, derives utility from consumption, $c_{t},$ in each period, with
standard additive separable life time utility
\[
U=\sum_{t=1}^{T}\beta^{t-1}s\left(  h_{t}\right)  u\left(  c_{t}\right)  ,
\]
where $\beta\in\left(  0,1\right)  $ is the discount factor. The instantaneous
utility function~$u\left(  c_{t}\right)  $ is assumed to be strictly concave,
and the probability to survive to the next period $s\left(  h_{t}\right)  $ is
assumed to be non-linear in the accumulated human (or health) capital stock
$h_{t}$. Income of the agent, $y_{t}$, consists of labor income which depends
on the amount of accumulated human capital $h_{t}$,
\[
y_{t}=wh_{t},
\]
where $w$ is the wage rate.

In each period the household faces the decision to consume $c_{t}$, to invest
in a risk-free financial asset, $a_{t}$, which earns (gross) interest $R$ and
to invest an amount $i_{t}$ in human capital, $h_{t}$. Human capital
depreciates at constant rate~$\delta$ and is produced by the production
function $f\left(  i\right)  $. We assume that $f_{i}>0$, $f_{ii}<0$ and that
the Inada conditions are satisfied, i.e., $\lim_{i_{t}\rightarrow0}%
f_{i}=\infty$ and~$\lim_{i_{t}\rightarrow\infty}f_{i}=0$. These conditions are
crucial because otherwise it could turn out to be optimal to invest in only
one asset. The other asset would be redundant and our problem would collapse
to a problem in one dimension. The human capital accumulation equation is
accordingly given by
\begin{equation}
h_{t+1}=(1-\delta)\left(  h_{t}+f\left(  i_{t}\right)  \right)  ,
\label{eq:hkaccum}%
\end{equation}
where $h_{0}$ is given.

Financial markets are imperfect and households are not allowed to hold
negative financial assets. The budget constraint writes as
\[
a_{t+1} = R(a_{t} + w h_{t} - c_{t} - i_{t}) \geq0.
\]


\paragraph{Recursive Formulation of the Household Problem}

The recursive formulation of the household problem is as follows:
\[
V_{t}(a_{t},h_{t})=\underset{c_{t},i_{t},a_{t+1},h_{t+1}}{\max}\left\{
u(c_{t})+\beta s\left(  h_{t+1}\right)  V_{t+1}(a_{t+1},h_{t+1})\right\}
\]
subject to the constraints%
\begin{align}
a_{t+1}  &  =R\left(  a_{t}+wh_{t}-c_{t}-i_{t}\right) \nonumber\\
h_{t+1}  &  =\left(  1-\delta\right)  \left(  h_{t}+f\left(  i_{t}\right)
\right) \nonumber\\
a_{t+1}  &  \geq0\nonumber\\
h_{t+1}  &  >0. \label{eq:hinequ}%
\end{align}


\paragraph{Assumptions on Functional Forms}

For our numerical approach we assume that instantaneous utility has the CRRA
property with coefficient of relative risk aversion denoted by~$\theta$:
\[
u\left(  c_{t}\right)  =\frac{c_{t}^{1-\theta}-1}{1-\theta}.
\]


The human capital production function is%
\[
f\left(  i_{t}\right)  =\gamma\frac{i_{t}^{1-\alpha}-1}{1-\alpha}%
\]
for curvature parameter~$\alpha>0$. As to the functional form of the
per-period survival probability we follow~%
%TCIMACRO{\TeXButton{citeN{hall07}}{\citeN{hall07}} }%
%BeginExpansion
\citeN{hall07}
%EndExpansion
and assume that
\[
s\left(  h_{t}\right)  =1-\phi\frac{1}{1+h_{t}}\text{.}%
\]


We assume that the value function is strictly concave and unique maximizers
are continuous policy functions, cf.
%TCIMACRO{\TeXButton{\citeN{stokey89}}{\citeN{stokey89}}}%
%BeginExpansion
\citeN{stokey89}%
%EndExpansion
. It is well-known that these conditions may not hold in human capital models
because endogenous human capital formation may lead to non-convexities (value
functions may have concave and convex regions). Hence, first-order conditions
are generally necessary but not sufficient. In applications, one way to
accommodate this is to use methods developed in this paper at the calibration
stage of the model (where speed is an issue). Upon convergence, one can then
test for uniqueness by checking for alternative solutions by use of global
methods. To focus our analysis we do not further address these aspects here.

\paragraph{Solution}

The optimal solution is fully characterized by the following set of
first-order conditions and constraints:
\begin{subequations}
\label{eq:eqsyst}%
\begin{align}
c_{t}^{-\theta}  &  =\beta R\left[  1-\phi\frac{1}{1+h_{t+1}}\right]
\text{$V_{t+1_{a}}$}\left(  a_{t+1},h_{t+1}\right) \label{Foc1}\\
i_{t}^{-\alpha}  &  =\frac{R}{\left(  1-\delta\right)  }\frac
{\text{$V_{t+1_{a}}$}\left(  a_{t+1},h_{t+1}\right)  }{\frac{\phi}{\left[
1+h_{t+1}-\phi\right]  \left[  1+h_{t+1}\right]  }V_{t+1}\left(
a_{t+1},h_{t+1}\right)  +\text{$V_{t+1_{h}}$}\left(  a_{t+1},h_{t+1}\right)
}\label{Foc2}\\
a_{t+1}  &  =R\left(  a_{t}+wh_{t}-c_{t}-i_{t}\right) \\
h_{t+1}  &  =\left(  1-\delta\right)  \left(  h_{t}+f\left(  i_{t}\right)
\right) \\
a_{t+1}  &  \geq0.
\end{align}
$V_{t_{a}}$ and $V_{t_{h}}$ are derivatives of the value function with respect
to financial assets and human capital, respectively. The first equation
relates today's consumption to consumption of tomorrow, whereas the second
equation relates costs and gains of investing in human capital. Notice that
constraint~(\ref{eq:hinequ}) can be dropped because of the lower Inada
condition of the human capital investment function~$f(i)$. Searching for the
solution of this model amounts to finding the four optimal policies for
consumption, $c_{t}\left(  \cdot,\cdot\right)  $, investment in human capital,
$i_{t}\left(  \cdot,\cdot\right)  $, next period's financial assets,
$a_{t+1}\left(  \cdot,\cdot\right)  $, and next period's human capital,
$h_{t+1}\left(  \cdot,\cdot\right)  $, as functions of the two endogenous
state variables, financial assets, $a_{t}$, and human capital, $h_{t}$, that
solve equation system~(\ref{eq:eqsyst}).

The envelope conditions are:
\end{subequations}
\begin{subequations}
\begin{align}
\text{$V_{t_{a}}$}\left(  a_{t},h_{t}\right)   &  =u_{c}=c_{t}^{-\theta
}\label{Envelope1}\\
\text{$V_{t_{h}}$}\left(  a_{t},h_{t}\right)   &  =\left(  w+\frac{1}{f_{i}%
}\right)  u_{c}=\left(  w+\frac{1}{i_{t}^{-\alpha}}\right)  c_{t}^{-\theta
}\text{.} \label{Envelope2}%
\end{align}
Using (\ref{Foc1}) together with (\ref{Envelope1}) gives the standard Euler
equation of consumption.\footnote{For derivation of~(\ref{eq:eqsyst}) and the
Envelope conditions see Appendix A.}

\subsection{Calibration}

We choose the same parameterization of the model for all solution methods
described in section~\ref{s:solmeth}. The coefficient of relative risk
aversion parameter is set to~$\theta=0.5$ to assure a positive value of life.
We set the time preference rate to $\rho=0.04$. In order to achieve incentives
to save in the finite horizon setting without introducing uncertainty we set
an interest rate of $R-1=0.05$. In the infinite horizon setting we set an
interest rate of $R-1=0.035$ smaller than $\rho$ to assure that financial
assets are bounded. For the depreciation rate of human capital we take
$\delta=0.05.$ The parameters of the human capital production function are
$\alpha=0.65$ and $\gamma=1.0$, respectively. The wage rate $w$ is set to
$0.1$. The survival rate parameter is $\phi=0.5$.

\section{Solution Methods}

\label{s:solmeth}

The main idea of all methods is to exploit the FOCs~(\ref{Foc1})
and~(\ref{Foc2}) to compute optimal policies at discrete points that
constitute a mesh in the state space. All three methods use the recursive
nature of the problem. Correspondingly, in the finite horizon version the
model is solved backwards from the last to the first period ($t=T,...,0$). In
the infinite horizon implementation the iteration continues until convergence
on policy functions.

Differences between methods arise because of different solution procedures to
the multi-dimensional nonlinear equation system~(\ref{eq:eqsyst}) and
different interpolation methods, respectively. The first algorithm (EXGM)
applies a multi-dimensional Quasi-Newton method. Standard interpolation
methods are used. The second algorithm (ENDGM) uses the method of endogenous
grid-points and thereby solves the system of equations~(\ref{eq:eqsyst})
analytically. It is accompanied by Delaunay interpolation. The third algorithm
(HEGM) combines the former two, i.e., it applies the method of endogenous
grid-points (and closed form solutions) in one dimension and uses a
one-dimensional Quasi-Newton method in the other dimension. As EXGM, HEGM
comes along with a standard interpolation procedure.

\subsection{Multi-Dimensional Root-Finding with Regular Interpolation (EXGM)}

The most direct approach to solve~(\ref{eq:eqsyst}) is to insert the
constraints into the FOCs and to rely on a numerical multi-dimensional
root-finding routine. Multi-dimensional solvers are necessary because $c$ and
$i$ show up on both sides of the respective equations in~(\ref{eq:eqsyst}). In
our application we use a Quasi-Newton method, more specifically Broyden's
method, cf.
%TCIMACRO{\TeXButton{\shortciteN{numerical90}}{\shortciteN{numerical90}}}%
%BeginExpansion
\shortciteN{numerical90}%
%EndExpansion
.

The implementation steps of EXGM are as follows:
\end{subequations}
\begin{enumerate}
\item To initialize EXGM predefine two grids, one for financial assets $a,$
$\mathcal{G}^{a}=\left\{  a^{1},a^{2},...,a^{K}\right\}  $ and one for human
capital $h$, $\mathcal{G}^{h}=\left\{  h^{1},h^{2},...,h^{J}\right\}  $ and
construct~$\mathcal{G}^{a,h}=\mathcal{G}^{a}\otimes\mathcal{G}^{h}$.

\item In period $T$, savings and investment in human capital are zero as both
assets are useless in period $T+1$\footnote{This rationale does not imply that
$h$ must be zero in period $T+1$ because human capital is---in contrast to
financial assets---inalienable.} and income is completely consumed for all
$\left(  a^{k},h^{j}\right)  \in\mathcal{G}^{a,h}$:
\begin{align*}
c_{T}\left(  \cdot,\cdot\right)   &  =a_{T}^{k}+wh_{T}^{j}\\
i_{T}\left(  \cdot,\cdot\right)   &  =0\text{.}%
\end{align*}
Using the above in equations~(\ref{Envelope1}) and~(\ref{Envelope2}) the value
function and its derivatives with respect to $a$ and $h$ in $T$ are
\begin{align*}
V_{T}\left(  a_{T}^{k},h_{T}^{j}\right)   &  =\frac{1}{1-\theta}\left[
c_{T}^{k,j}\right]  ^{1-\theta}\\
\text{$V_{T_{a}}$}\left(  a_{T}^{k},h_{T}^{j}\right)   &  =\left[  c_{T}%
^{k,j}\right]  ^{-\theta}\\
\text{$V_{T_{h}}$}\left(  a_{T}^{k},h_{T}^{j}\right)   &  =\left[  w+ \left(
i_{T}^{k,j}\right)  ^{\alpha}\right]  \left[  c_{T}^{k,j}\right]  ^{-\theta
}=w\left[  c_{T}^{k,j}\right]  ^{-\theta}\text{.}%
\end{align*}


\item Given functions $V_{t+1}$, $V_{t+1_{a}}$ and $V_{t+1_{h}}$ from the
previous step iterate backwards on $t=T-1,...,0$. For each $\left(  a_{t}%
^{k},h_{t}^{j}\right)  \in\mathcal{G}^{a,h}$ :

\begin{enumerate}
\item \label{EXGM interp}Solve the two-dimensional equation system
\begin{align*}
\left[  c_{t}^{k,j}\right]  ^{-\theta}  &  =\beta R\left(  1-\phi\frac
{1}{1+\left(  1-\delta\right)  \left(  h_{t}^{j}+\frac{1}{1-\alpha}\left(
i_{t}^{k,j}\right)  ^{1-\alpha}\right)  }\right) \\
&  \text{$V_{t+1_{a}}$}\left[  \overset{a_{t+1}^{k,j}}{\overbrace{R\left(
a_{t}^{k}+wh_{t}^{j}-c_{t}^{k,j}-i_{t}^{k,j}\right)  }},\overset{h_{t+1}%
^{k,j}}{\overbrace{\left(  1-\delta\right)  \left(  h_{t}^{j}+\frac
{1}{1-\alpha}\left(  i_{t}^{k,j}\right)  ^{1-\alpha}\right)  }}\right] \\
\left[  i_{t}^{k,j}\right]  ^{-\alpha}  &  =\frac{R}{\left(  1-\delta\right)
}\frac{\text{$V_{t+1_{a}}$}\left[  a_{t+1}^{k,j},h_{t+1}^{k,j}\right]  }%
{\frac{\phi}{\left(  1+h_{t+1}^{k,j}-\phi\right)  \left(  1+h_{t+1}%
^{k,j}\right)  }V_{t+1}\left[  a_{t+1}^{k,j},h_{t+1}^{k,j}\right]
+\text{$V_{t+1_{h}}$}\left[  a_{t+1}^{k,j},h_{t+1}^{k,j}\right]  }%
\end{align*}
for $c_{t}^{k,j}$ and $i_{t}^{k,j}$ using Broyden's method.

\item Save/Update both the value function and its derivatives%
\begin{align*}
V_{t}\left(  a_{t}^{k},h_{t}^{j}\right)   &  =\frac{1}{1-\theta}\left[
c_{t}^{k,j}\right]  ^{1-\theta}+\beta\left(  1-\phi\frac{1}{1+h_{t+1}^{k,j}%
}\right)  V_{t+1}(a_{t+1}^{k,j},h_{t+1}^{k,j})\\
\text{$V_{t+1_{a}}$}\left(  a_{t}^{k},h_{t}^{j}\right)   &  =\left[
c_{t}^{k,j}\right]  ^{-\theta}\\
\text{$V_{t+1_{h}}$}\left(  a_{t}^{k},h_{t}^{j}\right)   &  =\left[  w+
\left(  i_{T}^{k,j}\right)  ^{\alpha} \right]  \left[  c_{t}^{k,j}\right]
^{-\theta}\text{.}%
\end{align*}

\end{enumerate}
\end{enumerate}

Since EXGM requires to apply the solver for each point in~$\mathcal{G}^{a,h}$,
this procedure entails solving the multidimensional equation system $\left[
K\ast J\right]  $ times. Depending on the stopping criterion in the numerical
routine this could be either quite costly in terms of computing time or the
computed solutions suffer under low accuracy.\footnote{Similar to secant
methods in one dimension Broyden's algorithm only converges superlineraly
at~$O(n\log n)$ in the neighborhood of the root, cf.
%TCIMACRO{\TeXButton{\shortciteN{numerical77}}{\shortciteN{numerical77}}}%
%BeginExpansion
\shortciteN{numerical77}%
%EndExpansion
.} An additional shortcoming of EXGM compared to ENDGM and HEGM is that it
cannot exactly determine the region in which the borrowing constraint is
binding. Policy functions are imprecise at the kink. This may also cause
convergence problems. Furthermore, numerical methods often require fine tuning
so that stability of the solver is ascertained. For some parameter
constellations we in fact encountered instability problems in EXGM.

\paragraph{Interpolation on a Rectilinear Grid}

Step \ref{EXGM interp} requires evaluation of both the value function
$V_{t+1}$ and its derivatives , $V_{t+1_{a}}$ and $V_{t+1_{h}}$. We apply
piecewise linear interpolation. We determine interpolation nodes by the
concept \textquotedblleft grid square\textquotedblright, cf.
%TCIMACRO{\TeXButton{\shortciteN{numerical90}}{\shortciteN{numerical90}}}%
%BeginExpansion
\shortciteN{numerical90}%
%EndExpansion
. In order to apply this procedure it is necessary to have a rectilinear grid,
i.e., the state space has to be tessellated by rectangles.\footnote{Notice
that these rectangles do not necessarily have to be congruent to each other.}
In this case all grid-points in row $\mathcal{G}^{\bullet,j}$ have the same
value of $h^{j}$, and all grid-points in column $\mathcal{G}^{k,\bullet}$ have
the same value of $a^{k}$. The problem of locating a point in a
multi-dimensional grid is split up into several problems of locating the point
in one dimension. Within each dimension closest neighbors in the grid are
identified in about~$\text{log}_{2}N$ tries using bisection methods. Figure
\ref{Rectilinear_Grid} shows the location of interpolation nodes $[A;B;C;D]$
for point $X$ in a two-dimensional rectilinear grid.
%TCIMACRO{\TeXButton{B}{\begin{figure}[htb] \centering}}%
%BeginExpansion
\begin{figure}[htb] \centering
%EndExpansion
\caption{Rectilinear Grid}%
\begin{tabular}
[c]{p{15cm}}%
\\
\multicolumn{1}{c}{%
%TCIMACRO{\FRAME{itbpF}{9.0369cm}{6.0231cm}{0cm}{}{}{exo.eps}%
%{\special{ language "Scientific Word";  type "GRAPHIC";  display "USEDEF";
%valid_file "F";  width 9.0369cm;  height 6.0231cm;  depth 0cm;
%original-width 7.0845in;  original-height 5.1266in;  cropleft "0";
%croptop "0.9932";  cropright "1.0152";  cropbottom "0";
%filename 'Abbildungen/exo.eps';file-properties "XNPEU";}} }%
%BeginExpansion
\raisebox{-0cm}{\includegraphics[
trim=0.000000in 0.000000in -0.107685in 0.034861in,
natheight=5.126600in,
natwidth=7.084500in,
height=6.0231cm,
width=9.0369cm
]%
{Abbildungen/exo.eps}%
}
%EndExpansion
}\\
{\footnotesize {Interpolation on rectilinear grids. In any row locate the two
columns ($G^{k,\bullet}$ and $G^{k+1,\bullet}$) that form the most narrow
bracket of $a_{t+1}$. In any column locate the two rows ($G^{\bullet,j}$ and
$G^{\bullet,j+1}$) that form the most narrow bracket of $h_{t+1}$.
Interpolation nodes: $(k,j);(k+1,j);(k,j+1);(k+1,j+1)$.}}%
\end{tabular}
\label{Rectilinear_Grid}%
%TCIMACRO{\TeXButton{E}{\end{figure}}}%
%BeginExpansion
\end{figure}%
%EndExpansion


In EXGM, $\mathcal{G}^{a}\otimes\mathcal{G}^{h}$ is predetermined as a
rectilinear grid (in every iteration). After locating the nodes bilinear
interpolation of any function of~$F$---in our case the value function in~$t$
as well as its first derivatives with respect to~$a$ and~$h$--- at point X
requires computing $F\left(  X\right)  =\varphi_{A}F(A)+\varphi_{B}%
F(B)+\varphi_{C}F(C)+\varphi_{D}F(D)$ with the four basis functions $\varphi$
where%
\begin{align*}
\varphi_{A}  &  =p\ast q\\
\varphi_{B}  &  =\left(  1-p\right)  \ast q\\
\varphi_{C}  &  =p\ast\left(  1-q\right) \\
\varphi_{D}  &  =\left(  1-p\right)  \ast\left(  1-q\right)
\end{align*}
with $p=\frac{a_{X}-a_{A}}{a_{B}-a_{A}}$ and $q=\frac{h_{X}-h_{A}}{h_{C}%
-h_{A}}$, cf.
%TCIMACRO{\TeXButton{citeN{judd98}}{\citeN{judd98}}}%
%BeginExpansion
\citeN{judd98}%
%EndExpansion
.

\subsection{Analytical Solution with Delaunay Interpolation \newline(ENDGM)}

\label{ss:analendgm}

The above setting has a straightforward economic interpretation. Given an
exogenous state today $\left(  a_{t},h_{t}\right)  $ compute the endogenous
control variables $\left(  a_{t+1},h_{t+1}\right)  $. The main idea of ENDGM
is to redefine exogenous and endogenous objects in the numerical solution: the
control variable is taken as exogenous whereas today's state is taken
endogenous. In this sense, \textquotedblleft Endogenous Grid
Method\textquotedblright\ is a misnomer because computation is still based on
an exogenous grid, not for the state but for the control variables.\footnote{A
more appropriate name might be \textquotedblleft Endogenous State Grid
Method\textquotedblright.}

Computationally we first condition on a set of future endogenous state
variables, $\left(  a_{t+1},h_{t+1}\right)  $,---which are control variables
from the perspective of the current period---and exploit the system of FOCs to
determine the corresponding set of contemporaneous controls, $\left(
c_{t},i_{t}\right)  $. Second, we use the budget constraint and the law of
motion for human capital to get the according current set of endogenous state
variables, $\left(  a_{t},h_{t}\right)  $. Precisely, the implementation steps
are as follows:

\begin{enumerate}
\item To initialize ENDGM predefine two grids, one for gross savings $s$,
$G^{s}\equiv\left\{  s^{1},s^{2},...,s^{K}\right\}  $ which is defined as
$s=a_{t}+wh_{t}-c_{t}-i_{t}=\frac{a_{t+1}}{R}$ and one for human capital $z$,
$G^{z}\equiv\left\{  z^{1},z^{2},...,z^{J}\right\}  $ which is defined as
$z=h_{t}+f\left(  i_{t}\right)  =\frac{h_{t+1}}{1-\delta}$ and form
$\mathcal{G}^{s,z}=\mathcal{G}^{s}\otimes\mathcal{G}^{z}$

\item Define $\mathcal{G}^{a,h}=\mathcal{G}^{a}\otimes\mathcal{G}^{h}$ for
$T.$ In period $T$, as in EXGM,
\begin{align*}
c_{T}\left(  \cdot,\cdot\right)   &  =a_{T}^{k,j}+wh_{T}^{k,j}\\
i_{T}\left(  \cdot,\cdot\right)   &  =0
\end{align*}
for all $\left(  a^{k,j},h^{k,j}\right)  \in\mathcal{G}^{a,h}$ and
\begin{align*}
V_{T}\left(  a_{T}^{k,j},h_{T}^{k,j}\right)   &  =\frac{1}{1-\theta}\left[
c_{T}^{k,j}\right]  ^{1-\theta}\\
\text{$V_{T_{a}}$}\left(  a_{T}^{k,j},h_{T}^{k,j}\right)   &  =\left[
c_{T}^{k,j}\right]  ^{-\theta}\\
\text{$V_{T_{h}}$}\left(  a_{T}^{k,j},h_{T}^{k,j}\right)   &  =\left[  w+
\left(  i_{T}^{k,j}\right)  ^{\alpha}\right]  \left[  c_{T}^{k,j}\right]
^{-\theta}=w\left[  c_{T}^{k,j}\right]  ^{-\theta}\text{.}%
\end{align*}


\item Iterate backwards from~$t=T-1,...,0$. For each $\left(  s^{k}%
,z^{j}\right)  \in\mathcal{G}^{s,z}$:

\begin{enumerate}
\item Compute $a_{t+1}^{k}$ and $h_{t+1}^{j}$%
\begin{align*}
a_{t+1}^{k}  &  =Rs^{k},\\
h_{t+1}^{j}  &  =\left(  1-\delta\right)  z^{j}\text{.}%
\end{align*}


\item Given $V_{t+1}$, $V_{t+1_{a}}$ and $V_{t+1_{h}}$ from the previous
iteration step, interpolate the value function and its derivatives at $\left(
a_{t+1}^{k},h_{t+1}^{j}\right)  $ to get $V_{t+1}\left(  a_{t+1}^{k}%
,h_{t+1}^{j}\right)  $,\newline$V_{t+1_{a}}\left(  a_{t+1}^{k},h_{t+1}%
^{j}\right)  $ and $V_{t+1_{h}}\left(  a_{t+1}^{k},h_{t+1}^{j}\right)  $ using
Delaunay interpolation (see below).

\item \label{Advantage ENDGM}Compute $c_{t}^{k,j}$ and $i_{t}^{k,j}$%
\begin{align*}
c_{t}^{k,j}  &  =\left[  \beta R\left(  1-\phi\frac{1}{1+\left(
1-\delta\right)  z^{j}}\right)  \text{$V_{t+1_{a}}$}\left[  \overset
{a_{t+1}^{k}}{\overbrace{Rs^{k}}},\overset{h_{t+1}^{j}}{\overbrace{\left(
1-\delta\right)  z^{j}}}\right]  \right]  ^{-\frac{1}{\theta}},\\
i_{t}^{k,j}  &  =\left[  \frac{R}{\left(  1-\delta\right)  }\frac{V_{t+1_{a}%
}\left[  a_{t+1}^{k},h_{t+1}^{j}\right]  }{\frac{\phi}{\left(  1+h_{t+1}%
^{j}-\phi\right)  \left(  1+h_{t+1}^{j}\right)  }V_{t+1}\left[  a_{t+1}%
^{k},h_{t+1}^{j}\right]  +V_{t+1_{h}}\left[  a_{t+1}^{k},h_{t+1}^{j}\right]
}\right]  ^{-\frac{1}{\alpha}}\text{.}%
\end{align*}


\item Compute $a_{t}^{k,j}$ and $h_{t}^{k,j}$
\begin{align*}
h_{t}^{k,j}  &  =z^{j}-\frac{1}{1-\alpha} \left(  i_{t}^{k,j}\right)
^{1-\alpha}\\
a_{t}^{k,j}  &  =s^{k}-wh_{t}^{k,j}+c_{t}^{k,j}+i_{t}^{k,j}\text{.}%
\end{align*}


\item Save/Update both the value function and its derivatives%
\begin{align*}
V_{t}\left[  a_{t}^{k,j},h_{t}^{k,j}\right]   &  =\frac{1}{1-\theta}\left[
c_{t}^{k,j}\right]  ^{1-\theta}+\beta\left(  1-\phi\frac{1}{1+h_{t+1}^{j}%
}\right)  V_{t+1}\left[  a_{t+1}^{k},h_{t+1}^{j}\right] \\
V_{t_{a}}\left[  a_{t}^{k,j},h_{t}^{k,j}\right]   &  =\left[  c_{t}%
^{k,j}\right]  ^{-\theta}\\
V_{t_{h}}\left[  a_{t}^{k,j},h_{t}^{k,j}\right]   &  =\left[  w+ \left(
i_{T}^{k,j}\right)  ^{\alpha}\right]  \left[  c_{t}^{k,j}\right]  ^{-\theta
}\text{.}%
\end{align*}

\end{enumerate}
\end{enumerate}

The clear advantage of ENDGM compared to EXGM becomes obvious in step
\ref{Advantage ENDGM}. Because of the redefinition of~$a_{t+1}$ and~$h_{t+1}$
the system of FOCs can be solved for~$c_{t}$ and~$i_{t}$ analytically and
hence no numerical root-finder is needed. Furthermore, ENDGM provides, by
definition, an exact determination of the range of the borrowing constraint
and produces higher accuracy of the solution than EXGM in this region.
However, in contrast to the standard one-dimensional problem, the policy
function itself in the range where the borrowing constraint is binding does
not have a closed form solution in ENDGM.\footnote{In a standard
consumption-savings model with only one endogenous continuous state variable
the policy function is computed by linearly interpolating between the policy
at zero saving and the origin, cf.
%TCIMACRO{\TeXButton{citeN{carroll06}}{\citeN{carroll06}}}%
%BeginExpansion
\citeN{carroll06}%
%EndExpansion
.} In a multi-dimensional setting, the consumption and human capital
investment policy functions are not necessarily linear functions of current
cash on hand. To accommodate this we set exogenous grid-points in the region
of the borrowing constraint and use a one-dimensional root-finder in order to
compute the policy function. For more details see the appendix.

\begin{remark}
\label{rem:failureendgm} In contrast to EXGM, ENDGM is not a general method.
Suppose we were to adopt a general Ben-Porath human capital function, cf.
%TCIMACRO{\TeXButton{\citeN{Benporath67}}{\citeN{Benporath67}}}%
%BeginExpansion
\citeN{Benporath67}%
%EndExpansion
, in which the level of human capital directly affects the productivity of
human capital investments, i.e. we replace~$f(i)$ in
equation~(\ref{eq:hkaccum}) with~$f(h,i)$. ENDGM is no longer applicable in
such a formulation.
\end{remark}

\paragraph{Delaunay Interpolation}

In EXGM the grid is rectilinear by construction whereas in ENDGM the
endogenously computed grid~$\mathcal{G}^{a,h}$ is not. This constitutes the
main drawback of ENDGM because location of interpolation nodes is not obvious.
As illustrated in Figure~\ref{Irregular_Grid}, separating the
multi-dimensional problem into several one-dimensional problems is not
possible. In each row not just the value of~$a$ changes but also the value
of~$h$ so that the concept of bilinear interpolation in a square grid is not
applicable. ENDGM hence generates a situation where neighboring points in the
state space do not need to be neighboring elements in the grid matrix.
%TCIMACRO{\TeXButton{B}{\begin{figure}[htb] \centering}}%
%BeginExpansion
\begin{figure}[htb] \centering
%EndExpansion
\caption{Irregular Grid}%
\begin{tabular}
[c]{cc}
& \\
Panel (a) & Panel (b)\\%
%TCIMACRO{\FRAME{itbpF}{7.5278cm}{6.0231cm}{0cm}{}{}{endo_grid_1.eps}%
%{\special{ language "Scientific Word";  type "GRAPHIC";  display "USEDEF";
%valid_file "F";  width 7.5278cm;  height 6.0231cm;  depth 0cm;
%original-width 6.4091in;  original-height 5.5495in;  cropleft "0";
%croptop "0.9898";  cropright "1.0047";  cropbottom "0";
%filename 'Abbildungen/endo_grid_1.eps';file-properties "XNPEU";}} }%
%BeginExpansion
\raisebox{-0cm}{\includegraphics[
trim=0.000000in 0.000000in -0.030122in 0.056605in,
natheight=5.549500in,
natwidth=6.409100in,
height=6.0231cm,
width=7.5278cm
]%
{Abbildungen/endo_grid_1.eps}%
}
%EndExpansion
&
%TCIMACRO{\FRAME{itbpF}{7.5278cm}{6.0231cm}{0cm}{}{}{endo_grid_2.eps}%
%{\special{ language "Scientific Word";  type "GRAPHIC";  display "USEDEF";
%valid_file "F";  width 7.5278cm;  height 6.0231cm;  depth 0cm;
%original-width 6.3235in;  original-height 5.521in;  cropleft "0";
%croptop "1.0095";  cropright "0.9976";  cropbottom "0";
%filename 'Abbildungen/endo_grid_2.eps';file-properties "XNPEU";}} }%
%BeginExpansion
\raisebox{-0cm}{\includegraphics[
trim=0.000000in 0.000000in 0.015176in -0.052450in,
natheight=5.521000in,
natwidth=6.323500in,
height=6.0231cm,
width=7.5278cm
]%
{Abbildungen/endo_grid_2.eps}%
}
%EndExpansion
\\
\multicolumn{2}{p{15cm}}{{\footnotesize Interpolation on irregular grids.
Multidimensional interpolation cannot be separated into several
one-dimensional interpolations as the values of $a$ and $h$ change in each
column or row.}}%
\end{tabular}
\label{Irregular_Grid}%
%TCIMACRO{\TeXButton{E}{\end{figure}}}%
%BeginExpansion
\end{figure}%
%EndExpansion


The most common approach adopted in other scientific fields such as geometry
or geography to locate neighboring points in an irregular grid is the concept
of Delaunay triangulation and its related geometric construct, the Voronoi
diagram.\footnote{Recently, these methods have been introduced to Economics,
cf., e.g.,
%TCIMACRO{\TeXButton{\citeN{brumm10}}{\citeN{brumm10}}}%
%BeginExpansion
\citeN{brumm10}%
%EndExpansion
} We explain the geometric construction of the Voronoi diagram by use of
figure~\ref{Delaunay Triangulation}. The Voronoi diagram (polygon)---shown in
Panel~(a) of Figure~\ref{Delaunay Triangulation}---is the region of the state
space consisting of all points closer to gridpoint $P_{1}$ than to any other
gridpoint. The Voronoi diagram is obtained from the perpendicular bisectors of
the lines connecting neighboring points. Voronoi diagrams for all points form
a tessellation of the space, cf. Panel~(a). Edges of the Voronoi diagram are
all the points in the plane that are equidistant to the two nearest
grid-points, cf. Panel~(b). The Voronoi vertices are the points equidistant to
three grid-points, i.e., they are the center of circumcircles including the
three neighboring grid-points, cf. Panel~(c). Connecting these grid-points
constitutes the unique triangulation known as the Delaunay triangulation as
displayed in Panel~(d), cf.
%TCIMACRO{\TeXButton{\citeN{Baker99}}{\citeN{Baker99}}}%
%BeginExpansion
\citeN{Baker99}%
%EndExpansion
. The vertices of a triangle are the nearest neighbors of all points contained
in that triangle. These concepts can also be generalized to more than two dimensions.%

%TCIMACRO{\TeXButton{B}{\begin{figure}[htb] \centering}}%
%BeginExpansion
\begin{figure}[htb] \centering
%EndExpansion
\caption{The Voronoi Diagram}%
\begin{tabular}
[c]{cc}
& \\
Panel (a) & Panel (b)\\%
%TCIMACRO{\FRAME{itbpF}{6.0231cm}{6.0231cm}{0cm}{}{}{voronoi_2_shape.eps}%
%{\special{ language "Scientific Word";  type "GRAPHIC";  display "USEDEF";
%valid_file "F";  width 6.0231cm;  height 6.0231cm;  depth 0cm;
%original-width 6.7602in;  original-height 6.8312in;  cropleft "0";
%croptop "0.9988";  cropright "1.0135";  cropbottom "0";
%filename 'Abbildungen/Voronoi_2_shape.eps';file-properties "XNPEU";}} }%
%BeginExpansion
\raisebox{-0cm}{\includegraphics[
trim=0.000000in 0.000000in -0.091263in 0.008198in,
natheight=6.831200in,
natwidth=6.760200in,
height=6.0231cm,
width=6.0231cm
]%
{Abbildungen/Voronoi_2_shape.eps}%
}
%EndExpansion
&
%TCIMACRO{\FRAME{itbpF}{6.0231cm}{6.0231cm}{0cm}{}{}{voronoi_1_shape.eps}%
%{\special{ language "Scientific Word";  type "GRAPHIC";  display "USEDEF";
%valid_file "F";  width 6.0231cm;  height 6.0231cm;  depth 0cm;
%original-width 6.7602in;  original-height 6.8312in;  cropleft "0";
%croptop "0.9961";  cropright "0.9669";  cropbottom "0";
%filename 'Abbildungen/Voronoi_1_shape.eps';file-properties "XNPEU";}} }%
%BeginExpansion
\raisebox{-0cm}{\includegraphics[
trim=0.000000in 0.000000in 0.223763in 0.026642in,
natheight=6.831200in,
natwidth=6.760200in,
height=6.0231cm,
width=6.0231cm
]%
{Abbildungen/Voronoi_1_shape.eps}%
}
%EndExpansion
\\
Panel (c) & Panel (d)\\%
%TCIMACRO{\FRAME{itbpF}{6.0231cm}{6.0188cm}{0cm}{}{}{voronoi_3.eps}%
%{\special{ language "Scientific Word";  type "GRAPHIC";  display "USEDEF";
%valid_file "F";  width 6.0231cm;  height 6.0188cm;  depth 0cm;
%original-width 6.7602in;  original-height 6.8312in;  cropleft "0";
%croptop "1.0007";  cropright "0.9973";  cropbottom "0";
%filename 'Abbildungen/Voronoi_3.eps';file-properties "XNPEU";}} }%
%BeginExpansion
\raisebox{-0cm}{\includegraphics[
trim=0.000000in 0.000000in 0.018252in -0.004782in,
natheight=6.831200in,
natwidth=6.760200in,
height=6.0188cm,
width=6.0231cm
]%
{Abbildungen/Voronoi_3.eps}%
}
%EndExpansion
&
%TCIMACRO{\FRAME{itbpF}{6.0231cm}{6.0231cm}{0cm}{}{}{delaunay_!.eps}%
%{\special{ language "Scientific Word";  type "GRAPHIC";  display "USEDEF";
%valid_file "F";  width 6.0231cm;  height 6.0231cm;  depth 0cm;
%original-width 6.7602in;  original-height 6.8312in;  cropleft "0";
%croptop "1.0119";  cropright "0.9970";  cropbottom "0";
%filename 'Abbildungen/Delaunay_!.eps';file-properties "XNPEU";}} }%
%BeginExpansion
\raisebox{-0cm}{\includegraphics[
trim=0.000000in 0.000000in 0.020281in -0.081291in,
natheight=6.831200in,
natwidth=6.760200in,
height=6.0231cm,
width=6.0231cm
]%
{Abbildungen/Delaunay_!.eps}%
}
%EndExpansion
\\
\multicolumn{2}{p{15cm}}{{\footnotesize Panel (a): Generating the Voronoi
polygon: Edges are perpendicular bisectors of lines connecting neighboring
points. Panel (b): Several Voronoi tiles in mesh grid. Panel (c): Circle with
center at vertex includes three closest points. Panel (d): Delaunay
Triangulation: Vertices are nearest neighbors of all points within triangle.}}%
\end{tabular}
\label{Delaunay Triangulation}%
%TCIMACRO{\TeXButton{E}{\end{figure}}}%
%BeginExpansion
\end{figure}%
%EndExpansion


The computational implementation of a Delaunay triangulation is done by the
so-called randomized incremental algorithm, which we illustrate in
Figure~\ref{Delaunay Triangulation Computational}. It is incremental in the
sense that it adds points to the triangulation one at a time to maintain a
Delaunay triangulation at each stage. It is randomized in that points are
added in a random order which guarantees $O(N~log~N)$ expected time for the
algorithm, cf.~\shortciteN{numerical3rd}. To construct the Delaunay
triangulation for a given point set we initially have to add three
``fictitious'' points $\left[  \Theta_{1},\Theta_{2},\Theta_{3}\right]  $
forming a large starting triangle which encloses all ``real'' points, cf.
Panel (a) of Figure~\ref{Delaunay Triangulation Computational}. This is
necessary in order to ensure that added points lie within an existing
triangle. These ``fictitious'' points are deleted once the triangulation is
complete. In each following step of Delaunay triangulation a point from the
point set is added to the existing triangulation and connected to the vertices
of the enclosing triangle. We illustrate this step in Panel~(b) of the figure.
Consider the existing triangle~$P_{1},P_{2},P_{3}$ and a new point from the
point set,~$P_{5}$, which is not yet connected to other points.
Connecting~$P_{5}$ to~$P_{1}$, $P_{2}$ and $P_{3}$, respectively, gives rise
to three new triangles. Next, it is checked whether the newly created
triangles are ``legal'', i.e., whether the circumcircle of any triangle does
not contain any other point of the point set.\footnote{This principle is
derived from the definition that a triangulation fulfills the Delaunay
property if and only if the circumcircle of any triangle does not contain a
point in its interior, cf.
%TCIMACRO{\TeXButton{\shortciteN{berg08}}{\shortciteN{berg08}}}%
%BeginExpansion
\shortciteN{berg08}%
%EndExpansion
.} In our example, we first visit triangle~$P_{2},P_{3},P_{5}$ in Panel~(c).
As shown in the figure, the circumcircle contains point~$P_{4}$. Hence,
triangle~$P_{2},P_{3},P_{5}$ is not legal. Therefore, flip the edge opposite
of $P_{5}$ connecting~$P_{5}$ with~$P_{4}$. This operation creates two new
triangles, $P_{3},P_{4},P_{5}$ and~$P_{2},P_{4},P_{5}$, cf.~Panel~(d) of the
figure, which must be checked for legality. In our example, triangle~$P_{3}%
,P_{4},P_{5}$ is legal because the circumcircle does not contain other
existing points from the point set. The process is recursive and never wanders
away from any point~$P$ (point~$P_{5}$ in our example). The only edges that
can be made illegal by inserting a point $P$ are edges opposite $P$ (in
triangles with~$P$ as a vertex).\footnote{This procedure is described in
%TCIMACRO{\TeXButton{\citeN{numerical3rd}}{\shortciteN{numerical3rd}}}%
%BeginExpansion
\shortciteN{numerical3rd}%
%EndExpansion
. We use the numerical package geompack3 based on~%
%TCIMACRO{\TeXButton{\citeN{Joe91}}{\citeN{Joe91}} }%
%BeginExpansion
\citeN{Joe91}
%EndExpansion
for both the Delaunay triangulation and the ``visibility walk'', described
next.}%

%TCIMACRO{\TeXButton{B}{\begin{figure}[htb] \centering}}%
%BeginExpansion
\begin{figure}[htb] \centering
%EndExpansion
\caption{Incremental Algorithm}%
\begin{tabular}
[c]{cc}
& \\
Panel (a) & Panel (b)\\%
%TCIMACRO{\FRAME{itbpF}{6.0231cm}{6.0231cm}{0cm}{}{}{incremental_0.eps}%
%{\special{ language "Scientific Word";  type "GRAPHIC";  display "USEDEF";
%valid_file "F";  width 6.0231cm;  height 6.0231cm;  depth 0cm;
%original-width 6.8459in;  original-height 6.9436in;  cropleft "0";
%croptop "1.0047";  cropright "0.9970";  cropbottom "0";
%filename 'Abbildungen/Incremental_0.eps';file-properties "XNPEU";}} }%
%BeginExpansion
\raisebox{-0cm}{\includegraphics[
trim=0.000000in 0.000000in 0.020538in -0.032635in,
natheight=6.943600in,
natwidth=6.845900in,
height=6.0231cm,
width=6.0231cm
]%
{Abbildungen/Incremental_0.eps}%
}
%EndExpansion
&
%TCIMACRO{\FRAME{itbpF}{6.0226cm}{6.0226cm}{0cm}{}{}{incremental_1.eps}%
%{\special{ language "Scientific Word";  type "GRAPHIC";  display "USEDEF";
%valid_file "F";  width 6.0226cm;  height 6.0226cm;  depth 0cm;
%original-width 6.572in;  original-height 6.6658in;  cropleft "0";
%croptop "1.0031";  cropright "0.9997";  cropbottom "0";
%filename 'Abbildungen/Incremental_1.eps';file-properties "XNPEU";}} }%
%BeginExpansion
\raisebox{-0cm}{\includegraphics[
trim=0.000000in 0.000000in 0.001972in -0.020664in,
natheight=6.665800in,
natwidth=6.572000in,
height=6.0226cm,
width=6.0226cm
]%
{Abbildungen/Incremental_1.eps}%
}
%EndExpansion
\\
Panel (c) & Panel (d)\\
\multicolumn{1}{l}{%
%TCIMACRO{\FRAME{itbpF}{6.0226cm}{6.0226cm}{0cm}{}{}{incremental_2.eps}%
%{\special{ language "Scientific Word";  type "GRAPHIC";  display "USEDEF";
%valid_file "F";  width 6.0226cm;  height 6.0226cm;  depth 0cm;
%original-width 6.572in;  original-height 6.6658in;  cropleft "0";
%croptop "1.0031";  cropright "1.0066";  cropbottom "0";
%filename 'Abbildungen/Incremental_2.eps';file-properties "XNPEU";}} }%
%BeginExpansion
\raisebox{-0cm}{\includegraphics[
trim=0.000000in 0.000000in -0.043375in -0.020664in,
natheight=6.665800in,
natwidth=6.572000in,
height=6.0226cm,
width=6.0226cm
]%
{Abbildungen/Incremental_2.eps}%
}
%EndExpansion
} &
%TCIMACRO{\FRAME{itbpF}{6.0226cm}{6.0226cm}{0cm}{}{}{incremental_3.eps}%
%{\special{ language "Scientific Word";  type "GRAPHIC";  display "USEDEF";
%valid_file "F";  width 6.0226cm;  height 6.0226cm;  depth 0cm;
%original-width 6.572in;  original-height 6.6658in;  cropleft "0";
%croptop "1.0028";  cropright "0.9960";  cropbottom "0";
%filename 'Abbildungen/Incremental_3.eps';file-properties "XNPEU";}} }%
%BeginExpansion
\raisebox{-0cm}{\includegraphics[
trim=0.000000in 0.000000in 0.026288in -0.018664in,
natheight=6.665800in,
natwidth=6.572000in,
height=6.0226cm,
width=6.0226cm
]%
{Abbildungen/Incremental_3.eps}%
}
%EndExpansion
\\
\multicolumn{2}{p{15cm}}{{\footnotesize Panel (a): Three "fictional" points
added to constitute the first triangle which includes all "real" points of the
point set. Panel (b): Point added to existing Delaunay Triangulation and
connected to vertices of enclosing triangle. Panel (c): Circumcircle contains
a point. and is therefore illegal triangle. Panel (d): Circumcircle does not
contain any point and is therefore legal.}}%
\end{tabular}
\label{Delaunay Triangulation Computational}%
%TCIMACRO{\TeXButton{E}{\end{figure}}}%
%BeginExpansion
\end{figure}%
%EndExpansion


At interpolation, to locate a (query) point~$X$ in a given planar triangular
mesh we adopt a procedure referred to as visibility walk which is illustrated
in Figure~\ref{Visibility walk}. The search starts from an initial guess of a
triangle,~$\Delta_{1}$. Then, it is tested if the line supporting the first
edge~$e$ separates~$\Delta_{1}$ from the query point~$X$ which reduces to a
single operation test. If this is the case, the next triangle being visited is
the neighbor of~$\Delta_{1}$ through~$e$, $\Delta_{2}$. Otherwise the second
edge is tested in the same way. In case the test for the second edge also
fails then the third edge is tested. The failure of this third test means that
the goal has been reached. In Figure~\ref{Visibility walk}, this would be the
case at triangle $\Delta_{X}$ which contains $X$.\footnote{In non-Delaunay
triangulations, the visibility walk may fall into a cycle, whereas in Delaunay
triangulations the visibility walk always terminates, cf. ~%
%TCIMACRO{\TeXButton{\shortciteN{Devillers01}}{\shortciteN{Devillers01}}}%
%BeginExpansion
\shortciteN{Devillers01}%
%EndExpansion
.}
%TCIMACRO{\TeXButton{\shortciteN{Devillers01}}{\shortciteN{Devillers01}} }%
%BeginExpansion
\shortciteN{Devillers01}
%EndExpansion
find that performance of the visibility walk is better than other possible
algorithms. The location step for the visibility walk takes only $O\log\left(
N\right)  $ operations, cf.~%
%TCIMACRO{\TeXButton{\shortciteN{numerical3rd}}{\shortciteN{numerical3rd}}}%
%BeginExpansion
\shortciteN{numerical3rd}%
%EndExpansion
. The starting triangle may be arbitrary. However, an informed choice may
radically shorten the length of the walk. We accommodate this by initializing
the search with our solutions to grid-points visited previously.%

%TCIMACRO{\TeXButton{B}{\begin{figure}[htb] \centering}}%
%BeginExpansion
\begin{figure}[htb] \centering
%EndExpansion
\caption{Visibility Walk}%
\begin{tabular}
[c]{p{15cm}}%
\multicolumn{1}{c}{}\\
\multicolumn{1}{c}{%
%TCIMACRO{\FRAME{itbpF}{6.0275cm}{6.1cm}{0cm}{}{}{visibility_walk_2.eps}%
%{\special{ language "Scientific Word";  type "GRAPHIC";
%maintain-aspect-ratio TRUE;  display "USEDEF";  valid_file "F";
%width 6.0275cm;  height 6.1cm;  depth 0cm;  original-width 6.7602in;
%original-height 6.8312in;  cropleft "0";  croptop "1.0008";
%cropright "0.9989";  cropbottom "0";
%filename 'Abbildungen/visibility_walk_2.eps';file-properties "XNPEU";}} }%
%BeginExpansion
\raisebox{-0cm}{\includegraphics[
trim=0.000000in 0.000000in 0.007436in -0.005465in,
natheight=6.831200in,
natwidth=6.760200in,
height=6.1cm,
width=6.0275cm
]%
{Abbildungen/visibility_walk_2.eps}%
}
%EndExpansion
}\\
{\footnotesize Visibility walk in Delaunay triangulation - Locate triangle
$\Delta_{X}$ containing }$X${\footnotesize with initial guess $\Delta_{1}$. If
the line supporting $e$ separates $\Delta$ from }$X${\footnotesize , which
reduces to a single orientation test, then the next visited triangle is the
neighbor of $\Delta$ through $e$.}%
\end{tabular}
\label{Visibility walk}%
%TCIMACRO{\TeXButton{E}{\end{figure}}}%
%BeginExpansion
\end{figure}%
%EndExpansion


After locating the triangle we compute the normalized barycentric coordinates
(weights) of the query point $X$ with respect to the vertices $(A,B,C)$ of the
triangle $\Delta_{X}$%
\begin{align*}
\varphi_{A}  &  =\frac{\left(  a_{X}-a_{C}\right)  \left(  h_{B}-h_{C}\right)
+\left(  a_{C}-a_{B}\right)  \left(  h_{X}-h_{C}\right)  }{\left(  a_{A}%
-a_{C}\right)  \left(  h_{B}-h_{C}\right)  +\left(  a_{C}-a_{B}\right)
\left(  h_{A}-h_{C}\right)  }\\
\varphi_{B}  &  =\frac{\left(  a_{X}-a_{C}\right)  \left(  h_{C}-h_{A}\right)
+\left(  a_{A}-a_{C}\right)  \left(  h_{X}-h_{C}\right)  }{\left(  a_{A}%
-a_{C}\right)  \left(  h_{B}-h_{C}\right)  +\left(  a_{C}-a_{B}\right)
\left(  h_{A}-h_{C}\right)  }\\
\varphi_{C}  &  =1-\varphi_{A}-\varphi_{B}.
\end{align*}
Finally, the interpolated value of any function~$F$ at\ point $X$ is given as
the weighted average of the respective function values at the vertices,
\[
F\left(  X\right)  =\varphi_{A}F(A)+\varphi_{B}F(B)+\varphi_{C}F(C).
\]


\subsection{One-Dimensional Root-Finding with Hybrid Interpolation (HEGM)}

We next consider a hybrid method (HEGM) which combines EXGM and ENDGM.
Specifically, we use ENDGM in one dimension of the problem only. Hence, we
define one of the two state variables on an \textquotedblleft
endogenous\textquotedblright\ grid, whereas the other is on an
\textquotedblleft exogenous\textquotedblright\ grid. The algorithm proceeds in
three steps. In the first step, conditioning on control variable~$a_{t+1}$ and
current state~$h_{t}$, we exploit one of the two FOCs to derive one policy
function---in this setup investment in human capital,~$i_{t}$. In this step a
one-dimensional solver is required. To preserve comparability with the
previously described methods we choose Broyden's method.\footnote{Using
Brent's method instead turns out to slow down speed of HEGM.} In the second
step, policy function~$i_{t}$ is used to get the second endogenous state
variable,~$h_{t+1}$. Exploiting the second FOC we get the second policy
function,~$c_{t}$. In the third step, we compute the corresponding state
variable~$a_{t}$ from the budget constraint. The implementation steps are as follows:

\begin{enumerate}
\item To initialize HEGM predefine two grids, one for gross savings $s$,
$\mathcal{G}^{s}\equiv\left\{  s^{1},s^{2},...,s^{K}\right\}  $ and one for
human capital $h$, $\mathcal{G}^{h}\equiv\left\{  h^{1},h^{2},...,h^{J}%
\right\}  $ and form $\mathcal{G}^{s,h}=\mathcal{G}^{s}\otimes\mathcal{G}^{h}$

\item In period~$T$, define an initial guess for $\mathcal{G}^{a,h}%
=\mathcal{G}^{a}\otimes\mathcal{G}^{h}$ and compute
\begin{align*}
c_{T}\left(  \cdot,\cdot\right)   &  =a_{T}^{k,j}+wh_{T}^{j}\\
i_{T}\left(  \cdot,\cdot\right)   &  =0
\end{align*}
for all $\left(  a_{T}^{k,j},h_{T}^{j}\right)  \in\mathcal{G}^{a,h}$ and
\begin{align*}
V_{T}\left(  a_{T}^{k,j},h_{T}^{j}\right)   &  =\frac{1}{1-\theta}\left[
c_{T}^{k,j}\right]  ^{1-\theta}\\
\text{$V_{T_{a}}$}\left(  a_{T}^{k,j},h_{T}^{j}\right)   &  =\left[
c_{T}^{k,j}\right]  ^{-\theta}\\
\text{$V_{T_{h}}$}\left(  a_{T}^{k,j},h_{T}^{j}\right)   &  =\left[  w+\left(
i_{T}^{k,j}\right)  ^{\alpha}\right]  \left[  c_{T}^{k,j}\right]  ^{-\theta}.
\end{align*}


\item Given functions $V_{t+1}$, $V_{t+1_{a}}$ and $V_{t+1_{a}}$ from the
previous step iterate backwards on $t=T-1,...,0$. For each\ $\left(
s^{k},h^{j}\right)  \in\mathcal{G}^{s,h}$:

\begin{enumerate}
\item \label{HEGM solver}Solve the one-dimensional equation system for
$i_{t}^{k,j}$
\[
i_{t}^{k,j}=\left[  \frac{R}{\left(  1-\delta\right)  }\frac{\text{$V_{t+1_{a}%
} $}\left[  \overset{a_{t+1}^{k}}{\overbrace{Rs^{k}}},\overset{h_{t+1}^{k,j}%
}{\overbrace{\left(  1-\delta\right)  \left(  h_{t}^{j}+\frac{1}{1-\alpha
}\left(  i_{t}^{k,j}\right)  ^{1-\alpha}\right)  }}\right]  }{\frac{\phi
}{\left(  1+h_{t+1}^{k,j}-\phi\right)  \left(  1+h_{t+1}^{k,j}\right)
}V_{t+1}\left[  a_{t+1}^{k},h_{t+1}^{k,j}\right]  +\text{$V_{t+1_{h}}$}\left[
a_{t+1}^{k},h_{t+1}^{k,j}\right]  }\right]  ^{-\frac{1}{\alpha}}%
\]
using Broyden's method.\ This includes several computations of $a_{t+1}^{k}$
and $h_{t+1}^{k,j}$ and hybrid interpolations---described below---on $V_{t+1}
$, $V_{t+1_{a}}$and $V_{t+1_{h}}$.

\item Use $i_{t}^{k,j}$ to compute
\[
h_{t+1}^{k,j}=\left(  1-\delta\right)  \left(  h_{t}^{j}+\frac{1}{1-\alpha
}\left(  i_{t}^{k,j}\right)  ^{1-\alpha}\right)
\]
and next compute $c_{t}^{k,j}$ as
\[
c_{t}^{k,j}=\left(  \beta R\left(  1-\phi\frac{1}{1+h_{t+1}^{k,j}}\right)
\text{$V_{t+1_{a}}$}\left[  Rs^{k},h_{t+1}^{k,j}\right]  \right)  ^{-\frac
{1}{\theta}}.
\]


\item Compute $a_{t}^{k,j}$ from the budget constraint, hence
\[
a_{t}^{k,j}=s^{k}-wh_{t}^{j}+c_{t}^{k,j}+i_{t}^{k,j}.
\]


\item Save/Update both the value function and its derivatives%
\begin{align*}
V_{t}\left(  a_{t}^{k,j},h_{t}^{j}\right)   &  =\frac{1}{1-\theta}\left[
c_{t}^{k,j}\right]  ^{1-\theta}+\beta\left(  1-\phi\frac{1}{1+h_{t+1}^{k,j}%
}\right)  V_{t+1}(a_{t+1}^{k},h_{t+1}^{k,j})\\
\text{$V_{t_{a}}$}\left(  a_{t}^{k,j},h_{t}^{j}\right)   &  =\left[
c_{t}^{k,j}\right]  ^{-\theta}\\
\text{$V_{t_{h}}$}\left(  a_{t}^{k,j},h_{t}^{j}\right)   &  =\left[  w+\left(
i_{T}^{k,j}\right)  ^{\alpha}\right]  \left[  c_{t}^{k,j}\right]  ^{-\theta}.
\end{align*}

\end{enumerate}
\end{enumerate}

As EXGM, HEGM requires to run a numerical solver~$\left[  K\ast J\right]  $
times. However, computational burden is alleviated by reducing complexity of
the equation system. Furthermore, as in ENDGM, it is possible to exactly
determine the range of the borrowing constraint. In contrast to ENDGM in two
dimensions, there is no need for a complex interpolation method.

\begin{remark}
\label{rem:failurehegm} As ENDGM, HEGM is not a general method. Suppose that
consumption has an additional effect on human capital. Correspondingly
rewrite~(\ref{eq:hkaccum}) to~$h_{t+1}=\left(  1-\delta\right)  $ $\left(
h_{t}+f(i_{t})-g\left(  c_{t}\right)  \right)  $ to the effect that both
controls~$c_{t}$ and~$i_{t}$ appear on both sides of the equation system even
after applying the reformulation of endogenous states. This renders HEGM
inapplicable.\footnote{%
%TCIMACRO{\TeXButton{\citeN{Hintermaier10}}{\citeN{Hintermaier10}} }%
%BeginExpansion
\citeN{Hintermaier10}
%EndExpansion
show a potential way to solve this specific problem by a different kind of
HEGM.
%TCIMACRO{\TeXButton{\citeANP{Hintermaier10}}{\citeANP{Hintermaier10}} }%
%BeginExpansion
\citeANP{Hintermaier10}
%EndExpansion
replace the numerical solver in step~\ref{HEGM solver} with an additional
outer loop over a guess for a future endogenous state. In finite horizon
models, this procedure requires an exact knowledge of the state in the last
period. In a durable goods model,
%TCIMACRO{\TeXButton{\citeANP{Hintermaier10}}{\citeANP{Hintermaier10}} }%
%BeginExpansion
\citeANP{Hintermaier10}
%EndExpansion
set both states to zero in the last period T+1. In a human capital model such
as ours such an assumption is however invalid. It can only be assumed that
optimal investment in human capital is zero in the last period but not the
human capital stock itself.}
\end{remark}

\paragraph{Hybrid Interpolation}

Hybrid interpolation, illustrated in Figure~\ref{Hybrid interpolation}, is
defined on a curvilinear grid where one dimension is being held constant. To
locate any query point $X$ hybrid interpolation proceeds in three steps.
First, in the dimension of the exogenous grid (current state~$h_{t}$) find the
most narrow bracket of $h_{t+1}$ and compute the weights according to the
relative distance to these grid-points. Second, in both rows, find those
grid-points that form the most narrow bracket of~$a_{t+1}$ and compute the
according weights. Third, interpolation of any function of~$F$ at\ point $X$
requires computing $F\left(  X\right)  =\varphi_{A}F(A)+\varphi_{B}%
F(B)+\varphi_{C}F(C)+\varphi_{D}F(D)$ with the four basis functions $\varphi$
where
\begin{align*}
\varphi_{A}  &  =p\ast q\\
\varphi_{B}  &  =\left(  1-p\right)  \ast q\\
\varphi_{C}  &  =r\ast\left(  1-q\right) \\
\varphi_{D}  &  =\left(  1-r\right)  \ast\left(  1-q\right)
\end{align*}
with $p=\frac{a_{X}-a_{A}}{a_{B}-a_{A}}$, $r=\frac{a_{X}-a_{C}}{a_{D}-a_{C}}$
and $q=\frac{h_{X}-h_{C}}{h_{C}-h_{A}}$. Thus, HEGM reduces complexity of the
problem without involving advanced interpolation procedures.%

%TCIMACRO{\TeXButton{B}{\begin{figure}[htb] \centering}}%
%BeginExpansion
\begin{figure}[htb] \centering
%EndExpansion
\caption{Hybrid Interpolation}%
\begin{tabular}
[c]{p{15cm}}%
\multicolumn{1}{c}{}\\
\multicolumn{1}{c}{%
%TCIMACRO{\FRAME{itbpF}{9.0325cm}{6.0231cm}{0cm}{}{}{hybrid_3.eps}%
%{\special{ language "Scientific Word";  type "GRAPHIC";  display "USEDEF";
%valid_file "F";  width 9.0325cm;  height 6.0231cm;  depth 0cm;
%original-width 7.9009in;  original-height 5.5495in;  cropleft "0";
%croptop "1.0013";  cropright "0.9950";  cropbottom "0";
%filename 'Abbildungen/hybrid_3.eps';file-properties "XNPEU";}} }%
%BeginExpansion
\raisebox{-0cm}{\includegraphics[
trim=0.000000in 0.000000in 0.039504in -0.007214in,
natheight=5.549500in,
natwidth=7.900900in,
height=6.0231cm,
width=9.0325cm
]%
{Abbildungen/hybrid_3.eps}%
}
%EndExpansion
}\\
{\footnotesize Hybrid Interpolation. First, in the exogenous dimension, locate
the two rows $G^{\bullet,j}$ and $G^{\bullet,j+1}$ that form the most narrow
bracket of $h_{t+1}$. Second, locate in these two rows the grid-points that
form the most narrow bracket of $a_{t+1}$. Interpolation nodes: $(k,j)$;
$(k,j+1)$; $(l,j+1)$; $(l+1,j+1)$.}%
\end{tabular}
\label{Hybrid interpolation}%
%TCIMACRO{\TeXButton{E}{\end{figure}}}%
%BeginExpansion
\end{figure}%
%EndExpansion


\section{Results}

We present results separately for the finite and infinite horizon versions of
our model. Throughout, we use triple exponential grids for $a$, $h$, $s$,
$z$,\ respectively. We set the range of grid $G_{s}$ to $\left[  0,300\right]
$ and the range of $G_{z}$ to $\left[  1,300\right]  $. The according grids
$G_{a}$ and $G_{h}$ are adjusted to cover the corresponding range of the state space.

\subsection{Finite Horizon}

We iterate over $T=100$ time periods. Computational speed of the respective
algorithms is measured in seconds. Evaluation of accuracy of the solution is
done by applying normalized Euler equation errors, cf.~\citeN{judd92}, as has
become standard in the literature, cf., e.g.,~%
%TCIMACRO{\TeXButton{\citeN{Santos2000}}{\citeN{Santos2000}} }%
%BeginExpansion
\citeN{Santos2000}
%EndExpansion
and~%
%TCIMACRO{\TeXButton{\citeN{Barillas07}}{\citeN{Barillas07}}}%
%BeginExpansion
\citeN{Barillas07}%
%EndExpansion
. In our approach we get the Euler equation errors $e_{1}$ and $e_{2}$ by
using the respective envelope conditions and combine them with the FOCs to
get:
\begin{align*}
e_{1}  &  =1-\frac{\left[  Rs(h_{t+1})\beta\left(  c_{t+1}\right)  ^{-\theta
}\right]  ^{-\frac{1}{\theta}}}{c_{t}},\\
e_{2}  &  =1-\frac{\left[  \frac{R}{\left(  1-\delta\right)  }\left(
\frac{s_{h}(h_{t+1})V_{t+1}}{s(h_{t+1})\left(  c_{t+1}\right)  ^{-\theta}%
}+w+\frac{1}{\left(  i_{t+1}\right)  ^{-\alpha}}\right)  ^{-1}\right]
^{-\frac{1}{\alpha}}}{i_{t}}.
\end{align*}
This error is a dimension free quantity expressing the optimization error as a
fraction of current consumption. An error of $e_{1}=10^{-3}$, for instance,
means the household would make a \$1 mistake for each \$1000 spent, cf.
%TCIMACRO{\TeXButton{\citeN{aruoba06}}{\citeN{aruoba06}}}%
%BeginExpansion
\citeN{aruoba06}%
%EndExpansion
. These errors are expressed in units of base 10-logarithm which means that
$-4$ is an error of $0.0001$. To compare these methods in terms of accuracy we
simulate 80 life-cycles profiles. Initial assets are set to $a_{0}=10$ whereas
initial human capital $h_{0}$ differs in the range $\{50,70\}$. For each age
we compute $e_{1}$ and $e_{2}$.

Averages and maximum errors are provided in table \ref{results_finite}. Both
are of similar magnitudes across algorithms. To evaluate the relative
performance of the different algorithms, we can therefore further concentrate
on comparison speed only.

Table~\ref{results_finite} shows computing times for EXGM, ENDGM and HEGM for
different numbers of grid-points. As can further be seen in Panel~(a)
of~Figure~\ref{graph_finte} EXGM is outperformed by both ENDGM and HEGM.
Furthermore, we experience numerical stability problems with EXGM, especially
for large number of grid-points.\footnote{Similar problems are documented by~%
%TCIMACRO{\TeXButton{\citeN{Hintermaier10}}{\citeN{Hintermaier10}}}%
%BeginExpansion
\citeN{Hintermaier10}%
%EndExpansion
.} Handling the applied numerical routines becomes cumbersome and finding a
solution is not guaranteed. If possible, EXGM should therefore be avoided in
higher dimensional problems.

Panel~(b) of Figure~\ref{graph_finte} shows that ENDGM has a relative
advantage in comparison to HEGM in solving the model with a relatively small
number of grid-points. At a grid size of~$10^{2}$, ENDGM is almost two times
faster than HEGM. For solving the model with a higher number of grid-points,
however, HEGM is advantageous. At a grid size of~$500^{2}$ HEGM is two times
faster than ENDGM. In our setting the break-even point between both algorithms
is at a number of~$150^{2}$ grid-points and a computing time of~$5.5s.$ As can
be seen from table \ref{results_finite}, for a standard choice of~$20$ to~$40$
grid-points in each dimension, ENDGM is $\frac{0.4992}{0.3120}\approx1.6$
to~$\frac{0.1404}{0.078}\approx1.8$ times faster than HEGM and $\frac
{0.234}{0.078}\approx3$ to~$\frac{1.0452}{0.3120}\approx3.4$ times faster than EXGM.%

%TCIMACRO{\TeXButton{B}{\begin{table}[htb] \centering}}%
%BeginExpansion
\begin{table}[htb] \centering
%EndExpansion
\caption{Finite Horizon Model: Speed}%
\begin{tabular}
[c]{p{4cm}p{2cm}p{4cm}p{4cm}}%
\multicolumn{1}{l}{} &  & \multicolumn{1}{l}{} & \multicolumn{1}{l}{}\\
Number of Gridpoints for $(a,h)$ & \multicolumn{1}{||p{2cm}}{Time (seconds)} &
\multicolumn{1}{|p{4cm}}{Maximum Euler error for $\left(  c;i\right)  $} &
\multicolumn{1}{|p{4cm}}{Average Euler error for $\left(  c;i\right)  $%
}\\\hline
ENDGM & \multicolumn{1}{||p{2cm}}{} & \multicolumn{1}{|p{4cm}}{} &
\multicolumn{1}{|p{4cm}}{}\\
$\left(  20,20\right)  $ & \multicolumn{1}{||p{2cm}}{0.078} &
\multicolumn{1}{|p{4cm}}{$(-2.58;\ -1.33)$} &
\multicolumn{1}{|p{4cm}}{$(-3.52;\ -3.08)$}\\
$\left(  40,40\right)  $ & \multicolumn{1}{||p{2cm}}{0.3120} &
\multicolumn{1}{|p{4cm}}{$(-2.97;\ -1.32)$} &
\multicolumn{1}{|p{4cm}}{$(-4.28;\ -3.22)$}\\
$\left(  100,100\right)  $ & \multicolumn{1}{||p{2cm}}{2.1372} &
\multicolumn{1}{|p{4cm}}{$(-3.47;\ -1.33)$} &
\multicolumn{1}{|p{4cm}}{$(-5.15;\ -3.32)$}\\
$\left(  200,200\right)  $ & \multicolumn{1}{||p{2cm}}{10.6861} &
\multicolumn{1}{|p{4cm}}{$(-4.28;\ -1.33)$} &
\multicolumn{1}{|p{4cm}}{$(-5.75;\ -3.35)$}\\
& \multicolumn{1}{||p{2cm}}{} & \multicolumn{1}{|p{4cm}}{} &
\multicolumn{1}{|p{4cm}}{}\\\hline
HEGM & \multicolumn{1}{||p{2cm}}{} & \multicolumn{1}{|p{4cm}}{} &
\multicolumn{1}{|p{4cm}}{}\\
$\left(  20,20\right)  $ & \multicolumn{1}{||p{2cm}}{0.1404} &
\multicolumn{1}{|p{4cm}}{$(-2.76;\ -1.33)$} &
\multicolumn{1}{|p{4cm}}{$(-3.42;\ -2.76)$}\\
$\left(  40,40\right)  $ & \multicolumn{1}{||p{2cm}}{0.4992} &
\multicolumn{1}{|p{4cm}}{$(-3.00;\ -1.33)$} &
\multicolumn{1}{|p{4cm}}{$(-4.40;\ -3.13)$}\\
$\left(  100,100\right)  $ & \multicolumn{1}{||p{2cm}}{2.6832} &
\multicolumn{1}{|p{4cm}}{$(-3.44;\ -1.33)$} &
\multicolumn{1}{|p{4cm}}{$(-5.25;\ -3.31)$}\\
$\left(  200,200\right)  $ & \multicolumn{1}{||p{2cm}}{10.1245} &
\multicolumn{1}{|p{4cm}}{$(-3.95;\ -1.33)$} &
\multicolumn{1}{|p{4cm}}{$(-5.86;\ -3.35)$}\\
& \multicolumn{1}{||p{2cm}}{} & \multicolumn{1}{|p{4cm}}{} &
\multicolumn{1}{|p{4cm}}{}\\\hline
EXGM & \multicolumn{1}{||p{2cm}}{} & \multicolumn{1}{|p{4cm}}{} &
\multicolumn{1}{|p{4cm}}{}\\
$\left(  20,20\right)  $ & \multicolumn{1}{||p{2cm}}{0.234} &
\multicolumn{1}{|p{4cm}}{$(-2,81;\ -1,33)$} &
\multicolumn{1}{|p{4cm}}{$(-3.46;\ -2.75)$}\\
$\left(  40,40\right)  $ & \multicolumn{1}{||p{2cm}}{1.0452} &
\multicolumn{1}{|p{4cm}}{$(-3,00;\ -1,32)$} &
\multicolumn{1}{|p{4cm}}{$(-4.40;\ -3.12)$}\\
$\left(  100,100\right)  $ & \multicolumn{1}{||p{2cm}}{7.2384} &
\multicolumn{1}{|p{4cm}}{$(-3,44;\ -1,33)$} &
\multicolumn{1}{|p{4cm}}{$(-5.24;\ -3.31)$}\\
$\left(  200,200\right)  $ & \multicolumn{1}{||p{2cm}}{27.7214} &
\multicolumn{1}{|p{4cm}}{$(-3,94;\ -1,33)$} &
\multicolumn{1}{|p{4cm}}{$(-5.86;\ -3.35)$}\\\hline
\multicolumn{4}{p{14cm}}{{\footnotesize {Computing time for $T=100$ }and
resulting maximum and average Euler equation errors. Computing time is
reported in seconds and absolute errors in units of base-10 logarithms. }}%
\end{tabular}
\label{results_finite}%
%TCIMACRO{\TeXButton{E}{\end{table}}}%
%BeginExpansion
\end{table}%
%EndExpansion
%

%TCIMACRO{\TeXButton{B}{\begin{figure}[htb] \centering}}%
%BeginExpansion
\begin{figure}[htb] \centering
%EndExpansion
\caption{Finite Horizon Model: Speed}%
\begin{tabular}
[c]{cc}
& \\
Panel (a) & Panel (b)\\%
%TCIMACRO{\FRAME{itbpF}{7.0226cm}{6.0319cm}{0cm}{}{}{speed_all_finite.eps}%
%{\special{ language "Scientific Word";  type "GRAPHIC";  display "USEDEF";
%valid_file "F";  width 7.0226cm;  height 6.0319cm;  depth 0cm;
%original-width 5.7138in;  original-height 4.8066in;  cropleft "0";
%croptop "0.9992";  cropright "1.0010";  cropbottom "0";
%filename 'Abbildungen/speed_all_finite.eps';file-properties "XNPEU";}} }%
%BeginExpansion
\raisebox{-0cm}{\includegraphics[
trim=0.000000in 0.000000in -0.005714in 0.003845in,
natheight=4.806600in,
natwidth=5.713800in,
height=6.0319cm,
width=7.0226cm
]%
{Abbildungen/speed_all_finite.eps}%
}
%EndExpansion
&
%TCIMACRO{\FRAME{itbpF}{7.027cm}{6.0188cm}{0cm}{}{}{rel_endgm_hegm.eps}%
%{\special{ language "Scientific Word";  type "GRAPHIC";  display "USEDEF";
%valid_file "F";  width 7.027cm;  height 6.0188cm;  depth 0cm;
%original-width 5.7138in;  original-height 4.8066in;  cropleft "0";
%croptop "1.0007";  cropright "1.0053";  cropbottom "0";
%filename 'Abbildungen/rel_endgm_HEgm.eps';file-properties "XNPEU";}} }%
%BeginExpansion
\raisebox{-0cm}{\includegraphics[
trim=0.000000in 0.000000in -0.030283in -0.003365in,
natheight=4.806600in,
natwidth=5.713800in,
height=6.0188cm,
width=7.027cm
]%
{Abbildungen/rel_endgm_HEgm.eps}%
}
%EndExpansion
\\
\multicolumn{2}{p{15cm}}{{\footnotesize Panel (a): Computing time as a
function of grid-points in seconds (with equally many grid-points in both
dimensions). Solid line: computing time of EXGM; dotted line: computing time
of HEGM; dashed-dotted line: computing time of ENDGM. Panel (b): Ratio of
computing time of ENDGM to HEGM as a function of grid-points (with equally
many grid-points in both dimensions).}}%
\end{tabular}
\label{graph_finte}%
%TCIMACRO{\TeXButton{E}{\end{figure}}}%
%BeginExpansion
\end{figure}%
%EndExpansion


\subsection{Infinite horizon}

Algorithms are also comparable in the infinite horizon setting. In all
approaches, we make the same initial guesses for derivatives~$V_{0_{a}}$ and
$V_{0_{h}}$ and iterate until convergence on policy functions subject to
convergence criterion~$\varepsilon=10^{-6}$.\footnote{In our algorithms we
define $\tilde{c}=\left\{  \left\{  \tilde{c}^{k,j}\right\}  _{k=1}%
^{K}\right\}  _{j=1}^{J}$ and $\tilde{\imath}=\left\{  \left\{  \tilde{\imath
}^{k,j}\right\}  _{k=1}^{K}\right\}  _{j=1}^{J}$ and exit in iteration
$n\,$\ if $\sup|\tilde{c}-c^{n}|\leq\varepsilon\cup|\tilde{\imath}-i^{n}%
|\leq\varepsilon.$}

To compute Euler equation errors we simulate the response to a shock to
financial assets and health capital for 50 periods. We set the assets
$a_{0}=350$ and health in the range of $\{50,70\}$. We compute $e_{1}$ and
$e_{1}$ for the first 50 periods. Averages and maximum errors are provided in
table \ref{results_infinte}. As in the finite horizon setting, Euler equation
errors are of similar magnitudes across algorithms---which we also achieve by
appropriate settings of the respective numerical routines---so that we can
again further concentrate on comparison speed only.

In the infinite horizon setting, speed of ENDGM can be increased if the
Delaunay Triangulation is not constructed every iteration. Instead, we hold
the triangulation pattern fixed after a certain number of iterations---$100$
in our case.\footnote{It is necessary to assure that the endogenously computed
grid-points form a convex hull.} As in the finite horizon setting, ENDGM is
the fastest method for not too many grid-points and the comparative advantage
decreases for a higher number of grid-points. Both methods clearly dominate
EXGM in terms of speed. Similar to our findings in the finite horizon case,
for a standard choice of~$20$ to~$40$ grid-points in each dimension, ENDGM is
$\frac{0.41}{0.22}\approx1.9$ to~$\frac{1.39}{0.70}\approx2.0$ times faster
than HEGM and $\frac{0.69}{0.22}\approx3.1$ to~$\frac{3.03}{0.70}\approx4.3$
times faster than EXGM, cf. table \ref{results_infinte}. As in the finite
horizon setting EXGM faces severe problems in terms of stability for a high
number of grid-points.%

%TCIMACRO{\TeXButton{B}{\begin{figure}[htb] \centering}}%
%BeginExpansion
\begin{figure}[htb] \centering
%EndExpansion
\caption{Infinite Horizon Model: Speed}%
\begin{tabular}
[c]{cc}
& \\
Panel (a) & Panel (b)\\%
%TCIMACRO{\FRAME{itbpF}{6.9897cm}{5.8364cm}{0cm}{}{}%
%{speed_all_finite_smart.eps}{\special{ language "Scientific Word";
%type "GRAPHIC";  display "USEDEF";  valid_file "F";  width 6.9897cm;
%height 5.8364cm;  depth 0cm;  original-width 7.2463in;
%original-height 5.687in;  cropleft "0";  croptop "0.9981";
%cropright "1.0004";  cropbottom "0";
%filename 'Abbildungen/speed_all_finite_smart.eps';file-properties "XNPEU";}}
%}%
%BeginExpansion
\raisebox{-0cm}{\includegraphics[
trim=0.000000in 0.000000in -0.002898in 0.010805in,
natheight=5.687000in,
natwidth=7.246300in,
height=5.8364cm,
width=6.9897cm
]%
{Abbildungen/speed_all_finite_smart.eps}%
}
%EndExpansion
&
%TCIMACRO{\FRAME{itbpF}{6.9238cm}{5.8386cm}{0cm}{}{}%
%{rel_endgm_hegm_infinite_smart.eps}{\special{ language "Scientific Word";
%type "GRAPHIC";  display "USEDEF";  valid_file "F";  width 6.9238cm;
%height 5.8386cm;  depth 0cm;  original-width 7.2662in;
%original-height 5.8531in;  cropleft "0";  croptop "0.9984";
%cropright "0.9985";  cropbottom "0";
%filename 'Abbildungen/rel_endgm_HEgm_infinite_smart.eps';file-properties "XNPEU";}%
%} }%
%BeginExpansion
\raisebox{-0cm}{\includegraphics[
trim=0.000000in 0.000000in 0.010899in 0.009365in,
natheight=5.853100in,
natwidth=7.266200in,
height=5.8386cm,
width=6.9238cm
]%
{Abbildungen/rel_endgm_HEgm_infinite_smart.eps}%
}
%EndExpansion
\\
\multicolumn{2}{p{15cm}}{{\footnotesize Panel (a) Computing time to
convergence of policy functions (criterion $\varepsilon=10^{-6}$) as a
function of grid-points (with equally many grid-points in both dimensions).
Solid line: computing time of EXGM; dotted line: computing time of HEGM;
dashed-dotted line: computing time of ENDGM. Panel (b) Ratio of computing time
to convergence of ENDGM and HEGM as a function of grid-points (with equally
many grid-points in both dimensions).}}%
\end{tabular}
\label{graph_infinte}%
%TCIMACRO{\TeXButton{E}{\end{figure}}}%
%BeginExpansion
\end{figure}%
%EndExpansion
%

%TCIMACRO{\TeXButton{B}{\begin{table}[htb] \centering}}%
%BeginExpansion
\begin{table}[htb] \centering
%EndExpansion
\caption{Infinite Horizon Model: Performance Results}%
\begin{tabular}
[c]{p{2cm}p{2cm}p{2cm}p{3cm}p{3cm}}%
\multicolumn{1}{c}{} & \multicolumn{1}{c}{} & \multicolumn{1}{c}{} &
\multicolumn{1}{c}{} & \multicolumn{1}{c}{}\\
Number of Gridpoints for $(a,h)$ & \multicolumn{1}{||p{2cm}}{Time (seconds)} &
\multicolumn{1}{|p{2cm}}{Number of Iterations till convergence} &
\multicolumn{1}{|p{3cm}}{Maximum Euler error for $\left(  c;i\right)  $} &
\multicolumn{1}{|p{3cm}}{Average Euler error for $\left(  c;i\right)  $%
}\\\hline
ENDGM & \multicolumn{1}{||p{2cm}}{} & \multicolumn{1}{|p{2cm}}{} &
\multicolumn{1}{|p{3cm}}{} & \multicolumn{1}{|p{3cm}}{}\\
$\left(  20,20\right)  $ & \multicolumn{1}{||p{2cm}}{0.22} &
\multicolumn{1}{|p{2cm}}{307} & \multicolumn{1}{|p{3cm}}{$(-2.55;\ -1.62)$} &
\multicolumn{1}{|p{3cm}}{$(-3.16;\ -2.59)$}\\
$\left(  40,40\right)  $ & \multicolumn{1}{||p{2cm}}{0.70} &
\multicolumn{1}{|p{2cm}}{308} & \multicolumn{1}{|p{3cm}}{$(-3.12;\ -2.28)$} &
\multicolumn{1}{|p{3cm}}{$(-3.75;\ -3.17)$}\\
$\left(  100,100\right)  $ & \multicolumn{1}{||p{2cm}}{4.02} &
\multicolumn{1}{|p{2cm}}{308} & \multicolumn{1}{|p{3cm}}{$(-4.01;\ -2.82)$} &
\multicolumn{1}{|p{3cm}}{$(-4.89;\ -3.87)$}\\
$\left(  200,200\right)  $ & \multicolumn{1}{||p{2cm}}{16.99} &
\multicolumn{1}{|p{2cm}}{308} & \multicolumn{1}{|p{3cm}}{$(-4.37;\ -3.50)$} &
\multicolumn{1}{|p{3cm}}{$(-5.27;\ -4.39)$}\\
& \multicolumn{1}{||p{2cm}}{} &  &  & \\\hline
HEGM & \multicolumn{1}{||p{2cm}}{} & \multicolumn{1}{|p{2cm}}{} &
\multicolumn{1}{|p{3cm}}{} & \multicolumn{1}{|p{3cm}}{}\\
$\left(  20,20\right)  $ & \multicolumn{1}{||p{2cm}}{0.41} &
\multicolumn{1}{|p{2cm}}{318} & \multicolumn{1}{|p{3cm}}{$(-2.77;\ -1.96)$} &
\multicolumn{1}{|p{3cm}}{$(-3.41;\ -2.73)$}\\
$\left(  40,40\right)  $ & \multicolumn{1}{||p{2cm}}{1.39} &
\multicolumn{1}{|p{2cm}}{316} & \multicolumn{1}{|p{3cm}}{$(-3.20;\ -2.11)$} &
\multicolumn{1}{|p{3cm}}{$(-3.94;\ -3.17)$}\\
$\left(  100,100\right)  $ & \multicolumn{1}{||p{2cm}}{7.97} &
\multicolumn{1}{|p{2cm}}{320} & \multicolumn{1}{|p{3cm}}{$(-4.12;\ -2.76)$} &
\multicolumn{1}{|p{3cm}}{$(-4.94;\ -3.78)$}\\
$\left(  200,200\right)  $ & \multicolumn{1}{||p{2cm}}{31.17} &
\multicolumn{1}{|p{2cm}}{320} & \multicolumn{1}{|p{3cm}}{$(-4.64;\ -3.39)$} &
\multicolumn{1}{|p{3cm}}{$(-5.36;\ -4.27)$}\\
& \multicolumn{1}{||p{2cm}}{} &  &  & \\\hline
EXGM & \multicolumn{1}{||p{2cm}}{} & \multicolumn{1}{|p{2cm}}{} &
\multicolumn{1}{|p{3cm}}{} & \multicolumn{1}{|p{3cm}}{}\\
$\left(  20,20\right)  $ & \multicolumn{1}{||p{2cm}}{0.69} &
\multicolumn{1}{|p{2cm}}{346} & \multicolumn{1}{|p{3cm}}{$(-2,87;\ -1,96)$} &
\multicolumn{1}{|p{3cm}}{$(-3.53;\ -2.73)$}\\
$\left(  40,40\right)  $ & \multicolumn{1}{||p{2cm}}{3.03} &
\multicolumn{1}{|p{2cm}}{379} & \multicolumn{1}{|p{3cm}}{$(-3,30;\ -2,11)$} &
\multicolumn{1}{|p{3cm}}{$(-4.02;\ -3.16)$}\\
$\left(  100,100\right)  $ & \multicolumn{1}{||p{2cm}}{21.03} &
\multicolumn{1}{|p{2cm}}{407} & \multicolumn{1}{|p{3cm}}{$(-4,14;\ -2,76)$} &
\multicolumn{1}{|p{3cm}}{$(-4.93;\ -3.79)$}\\
$\left(  200,200\right)  $ & \multicolumn{1}{||p{2cm}}{86.94} &
\multicolumn{1}{|p{2cm}}{418} & \multicolumn{1}{|p{3cm}}{$(-4,66;\ -3,39)$} &
\multicolumn{1}{|p{3cm}}{$(-5.37;\ -4.27)$}\\\hline
\multicolumn{5}{p{14cm}}{{\footnotesize Computing time to convergence of
policy functions (criterion $\varepsilon=10^{-6}$) and resulting maximum and
average Euler equation errors. Computing time is reported in seconds and
absolute errors in units of base-10 logarithms. }}%
\end{tabular}
\label{results_infinte}%
%TCIMACRO{\TeXButton{E}{\end{table}}}%
%BeginExpansion
\end{table}%
%EndExpansion


\clearpage


\section{Conclusion}

We compare three numerical methods---the standard exogenous grid method
(EXGM), Carroll's method of endogenous grid-points (ENDGM), cf.
%TCIMACRO{\TeXButton{citeN{carroll06}}{\citeN{carroll06}}}%
%BeginExpansion
\citeN{carroll06}%
%EndExpansion
, and a hybrid method (HEGM)---to solve dynamic models with two continuous
state variables and occasionally binding borrowing constraints. To illustrate
and to evaluate these methods we develop a life-cycle consumption-savings
model with endogenous human capital formation. Evaluation of methods is based
on speed and accuracy in both finite and infinite horizon settings. We show
that applying the endogenous grid method in more than one dimension gives rise
to irregular grids. We apply Delaunay methods to interpolate on these
irregular grids. Despite this more complex interpolation, ENDGM and HEGM
clearly outperform EXGM. ENDGM dominates HEGM for small to medium sized
problems (in terms of number of grid-points) whereas HEGM dominates for a
large number of grid-points. For a standard choice of numbers of grid-points,
ENDGM is $1.6$ to~$1.8$ times faster than HEGM.

Two additional remarks on ENDGM and HEGM are in order. First, neither of the
two is a general method. Both are applicable only to specific problems at
hand. This requires restrictions on the model's specification and on
functional forms. Second, as HEGM uses analytical solutions in only one
dimension and standard numerical methods in others, it's relative advantage
decreases in the dimensionality of the problem.

In this paper, however, we restrict attention to two dimensional problems.
Extensions to higher dimensions are left for future research.

\clearpage\newpage

\section{Appendix}

\label{app:equations}

To derive first-order conditions, we focus on the infinite horizon setting. As
is standard in the literature, we denote next period values by
symbol~$^{\prime}$.

The dynamic version of the household problem reads as
\[
V(a,h)=\underset{c,i,a^{\prime},h^{\prime}}{\max}\left\{  u(c)+\beta s\left(
h^{\prime}\right)  V^{\prime}(a^{\prime},h^{\prime})\right\}
\]
subject to
\begin{align*}
a^{\prime}  &  =R\left(  a+wh-c-i\right) \\
h^{\prime}  &  =\left(  1-\delta\right)  \left(  h+f\left(  i\right)  \right)
\\
a^{\prime}  &  \geq0.
\end{align*}


Assigning multiplier $\mu$ to the borrowing constraint, the two first order
conditions with respect to~$c$ and~$k$ are:
\begin{equation}
\frac{\partial V\left(  a,h\right)  }{\partial c}=u_{c}-\beta s\left(
h^{\prime}\right)  V_{a}^{\prime}R-R\mu\overset{!}{=}0\;\Leftrightarrow
\;u_{c}-\beta s\left(  h^{\prime}\right)  RV_{a}^{\prime}=R\mu, \label{FOC1}%
\end{equation}%
\begin{align}
\frac{\partial V\left(  a,h\right)  }{\partial i}  &  =s_{h}\left(  h^{\prime
}\right)  \left(  1-\delta\right)  f_{i}\beta V^{\prime}+s\left(  h^{\prime
}\right)  \beta\left[  V_{a}^{\prime}\left(  -R\right)  +V_{h}^{\prime}\left(
1-\delta\right)  f_{i}\right]  -R\mu\overset{!}{=}0\nonumber\\
&  \Leftrightarrow s_{h}\left(  h^{\prime}\right)  \left(  1-\delta\right)
f_{i}\beta V^{\prime}+s\left(  h^{\prime}\right)  \beta\left[  V_{a}^{\prime
}\left(  -R\right)  +V_{h}^{\prime}\left(  1-\delta\right)  f_{i}\right]
=R\mu\label{FOC2}%
\end{align}


and%
\begin{align*}
a^{\prime}  &  \geq0\\
\mu &  \geq0\\
a^{\prime}\mu &  =0.
\end{align*}


In order to compute optimal policies we need to distinguish two cases.

\subsection*{Case 1: Interior Solution}

In the first case the borrowing constraint is not binding so that $\mu=0$.
This reduces the system of equations to%
\begin{align*}
u_{c}-\beta s\left(  h^{\prime}\right)  RV_{a}^{\prime}  &  =0\\
s_{h}\left(  h^{\prime}\right)  \left(  1-\delta\right)  f_{i}\beta V^{\prime
}+s\left(  h^{\prime}\right)  \beta\left[  V_{a}^{\prime}\left(  -R\right)
+V_{h}^{\prime}\left(  1-\delta\right)  f_{i}\right]   &  =0.
\end{align*}
Rearranging gives
\begin{align*}
u_{c}  &  =\beta s\left(  h^{\prime}\right)  V_{a}^{\prime}R\\
f_{i}  &  =\frac{R}{\left(  1-\delta\right)  }\frac{s\left(  h^{\prime
}\right)  V_{a}^{\prime}}{s_{h}\left(  h^{\prime}\right)  V^{\prime}+s\left(
h^{\prime}\right)  V_{h}^{\prime}}.
\end{align*}


\subsection*{Case 2: Corner Solution---Binding Borrowing Constraint}

In the second case the borrowing constraint is binding so that $a^{\prime}=0 $
and $\mu>0$. From~(\ref{FOC1}) and~(\ref{FOC2}) it then follows that
\begin{equation}
u_{c}=s_{h}\left(  h^{\prime}\right)  \beta V^{\prime}+s\left(  h^{\prime
}\right)  \beta V_{h}^{\prime}\left(  1-\delta\right)  f_{i} \label{eq:uc}%
\end{equation}
and%
\[
u_{c}=\beta\left(  1-\delta\right)  f_{i}\left[  s_{h}\left(  h^{\prime
}\right)  V^{\prime}+s\left(  h^{\prime}\right)  V_{h}^{\prime}\right]
\]%
\[
a^{\prime}=0\;\Leftrightarrow\;c=a+wh-i.
\]


Making use of our assumptions on functional forms, equation~(\ref{eq:uc})
reduces in EXGM and HEGM to
\begin{align*}
&  \left(  a+wh-i\right)  ^{-\theta}-\frac{1}{\left[  \left(  1-\delta\right)
\left(  h+\frac{i^{1-\alpha}}{1-\alpha}\right)  \right]  ^{2}}V^{\prime
}\left[  0,\left(  1-\delta\right)  \left(  h+\frac{i^{1-\alpha}}{1-\alpha
}\right)  \right]  \beta\left(  1-\delta\right)  i^{-\alpha}\\
&  -\left[  1-\frac{1}{\left[  \left(  1-\delta\right)  \left(  h+\frac
{i^{1-\alpha}}{1-\alpha}\right)  \right]  }\right]  \left[  V_{h}^{\prime
}\left[  0,\left(  1-\delta\right)  \left(  h+\frac{i^{1-\alpha}}{1-\alpha
}\right)  \right]  \right]  \beta\left(  1-\delta\right)  i^{-\alpha}=0
\end{align*}
and in ENDGM to
\begin{align*}
&  \left(  a+w\frac{h^{\prime}}{1-\delta}-i^{\alpha}-i\right)  ^{-\theta
}-\frac{1}{\left[  h^{\prime}\right]  ^{2}}\beta V^{\prime}\left[
0,h^{\prime}\right]  \left(  1-\delta\right)  i^{-\alpha}\\
&  -\left[  1-\frac{1}{1+h^{\prime}}\right]  \beta\left[  V_{h}^{\prime
}\left[  0,h^{\prime}\right]  \right]  \left(  1-\delta\right)  i^{-\alpha}=0.
\end{align*}
Observe that this equation is not linear in $i$. We therefore need to use a
numerical routine in the region where the borrowing constraint is binding also
for ENDGM, cf. our discussion in the main text in
Subsection~\ref{ss:analendgm}.

\qquad

In both cases---i.e. for interior solutions and for binding borrowing
constraints---the envelope conditions are
\begin{align*}
\frac{\partial V\left(  x,h\right)  }{\partial a}  &  \equiv V_{a}=\beta
V_{a}^{\prime}R+R\mu=u_{c}\\
\frac{\partial V\left(  x,h\right)  }{\partial h}  &  \equiv V_{h}\\
&  =\beta s_{h}\left(  h^{\prime}\right)  V(a^{\prime},h^{\prime})\left(
1-\delta\right)  +\beta s\left(  h^{\prime}\right)  V_{a}(a^{\prime}%
,h^{\prime})wR+\beta s\left(  h^{\prime}\right)  V_{h}(a^{\prime},h^{\prime
})\left(  1-\delta\right)  +R\mu\\
&  =\left(  w+\frac{1}{f_{i}}\right)  u_{c}.
\end{align*}
\newpage
\bibliographystyle{chicago}
\bibliography{comparison}

\end{document}